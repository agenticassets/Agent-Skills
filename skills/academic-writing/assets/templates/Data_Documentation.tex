% TEMPLATE NOTE: This is a generic template document for documenting data sources and methodology.
% This file can be compiled standalone for reference, or the content can be integrated into
% an Appendix section in Main.tex using % TEMPLATE NOTE: This is a generic template document for documenting data sources and methodology.
% This file can be compiled standalone for reference, or the content can be integrated into
% an Appendix section in Main.tex using % TEMPLATE NOTE: This is a generic template document for documenting data sources and methodology.
% This file can be compiled standalone for reference, or the content can be integrated into
% an Appendix section in Main.tex using % TEMPLATE NOTE: This is a generic template document for documenting data sources and methodology.
% This file can be compiled standalone for reference, or the content can be integrated into
% an Appendix section in Main.tex using \input{sections/Data_Documentation.tex}
%
% To adapt this template:
% 1. Replace [PLACEHOLDER] items with your specific data sources and variables
% 2. Update methodology sections with your actual analysis approach
% 3. Modify variable definitions to match your project requirements
% 4. For integration into Main.tex, remove \documentclass-\begin{document} and \end{document}

\documentclass[11pt]{article}
\usepackage[utf8]{inputenc}
\usepackage[english]{babel}
\usepackage{amsmath}
\usepackage{amssymb}
\usepackage{graphicx}
\usepackage[top=1in, bottom=1in, left=1in, right=1in]{geometry}
\usepackage{hyperref}
\usepackage{longtable}
\usepackage{fancyhdr}
\usepackage{setspace}
\usepackage{float} % For [H] placement of figures/tables

% Custom commands for consistency
\newcommand{\variablefontsize}{\fontsize{9pt}{11pt}\selectfont} % Adjust font size for variable definitions
\newcommand{\tablefontsize}{\fontsize{10pt}{12pt}\selectfont} % Adjust font size for tables

% Page style (no header/footer for simplicity, but can be customized)
\pagestyle{plain}

% Adjust paragraph spacing
\setlength{\parindent}{0pt}
\setlength{\parskip}{1em}

\begin{document}

\section*{Data Sources and Methodology Documentation}
\subsection*{[RESEARCH PROJECT TITLE]}

\subsubsection*{Overview}
This document describes the data sources, variable construction methods, and analytical methodology for the research project examining [RESEARCH QUESTION/HYPOTHESIS].
The documentation provides details on [PRIMARY DATA SOURCE(S)] and how [PRIMARY OUTCOME VARIABLE] is constructed and measured across the sample.

---

\section*{1. Data Sources}

\subsection*{1.1 Primary Data Sources}

\textbf{[Data Source 1: Primary Market/Firm Data]}
\begin{itemize}
    \item [Time frequency and coverage period: e.g., daily/monthly, YYYY-YYYY]
    \item [Key variables provided: e.g., returns, market cap, firm characteristics]
    \item [Additional variables: e.g., location, sector, size indicators]
    \item Used for: [Primary analysis purpose]
    \item Files: [Format and storage location specification]
\end{itemize}

\textbf{[Data Source 2: Secondary Financial Data]}
\begin{itemize}
    \item [Specific data type and frequency: e.g., quarterly accounting data, YYYY-YYYY]
    \item [Firm characteristics available: e.g., financial statements, accounting variables]
    \item [Geographic/identifying information: e.g., headquarters location]
    \item Used for: [Purpose: e.g., control variables, firm linking, sample construction]
\end{itemize}

\textbf{[Data Source 3: Optional Additional Financial Data]}
\begin{itemize}
    \item [Data type and coverage: e.g., analyst forecasts, YYYY-YYYY]
    \item [Specific measures: e.g., forecast dispersion, consensus estimates]
    \item [Coverage and timing: e.g., frequency of measurements, availability]
    \item Used for: [Purpose in analysis]
\end{itemize}

\textbf{[Data Source 4: Factor/Risk Data]}
\begin{itemize}
    \item [Type of factors or risk data: e.g., factor returns, benchmark data]
    \item [Frequency and coverage: e.g., daily returns, YYYY-YYYY]
    \item [Specific factors included: e.g., market, size, value, etc.]
    \item Used for: [Purpose: e.g., risk adjustment, robustness checks]
\end{itemize}

\subsection*{1.2 Treatment and Exogenous Variables}

\textbf{[Primary Treatment/Exposure Variable Source]}
\begin{itemize}
    \item [Description of treatment/exposure: e.g., geographic, temporal, firm-level]
    \item [Measurement level: e.g., county-level, firm-level, industry-level]
    \item [Temporal coverage and frequency: e.g., annual, quarterly]
    \item [Source and documentation: cite original data source]
\end{itemize}

\textbf{[Alternative Data Source: e.g., Additional Measures or Proxies]}
\begin{itemize}
    \item [Description and measurement approach]
    \item [How it relates to primary treatment variable]
    \item [Time periods and frequencies: YYYY-YYYY]
    \item [Aggregation method: e.g., geographic, temporal aggregation]
\end{itemize}

\subsection*{1.3 Control Variables and Demographic Data}

\textbf{[Demographic/Economic Control Data Source]}
\begin{itemize}
    \item [Geographic level: e.g., county-level, state-level]
    \item [Variables available: e.g., population, income, education, employment]
    \item [Frequency and coverage: YYYY-YYYY, annual/quarterly]
    \item [Linking method to analysis sample]
\end{itemize}

\textbf{[Additional Control Variables Source]}
\begin{itemize}
    \item [Type of variables: e.g., economic indicators, market conditions]
    \item [Specific measures: list key variables]
    \item [Coverage period and frequency]
\end{itemize}



---

\section*{2. Variable Construction Methodology}

\subsection*{2.1 Primary Outcome Variable}

\textbf{[Primary Outcome Variable: Name and Definition]}
\begin{itemize}
    \item \textbf{Description}: [Description of what this variable measures and its theoretical importance]
    \item \textbf{Measurement approach}: [How variable is calculated: formula, time-series method, etc.]
    \item \textbf{Data inputs}: [Which source data are used in construction]
    \item \textbf{Frequency}: [Unit of observation: daily, monthly, annual; firm-level, industry-level, etc.]
\end{itemize}

\textbf{Construction Process:}
\begin{enumerate}
    \item [Step 1: Data preparation or filtering]
    \item [Step 2: Raw variable calculation or extraction]
    \item [Step 3: Aggregation or transformation (if applicable)]
    \item [Step 4: Quality checks or validation]
    \item [Step 5: Linking to other datasets (if applicable)]
\end{enumerate}

\subsection*{2.2 Secondary Outcome or Risk Variables}

\textbf{[Secondary Variable Name: Alternative Measure]}
\begin{itemize}
    \item \textbf{Description}: [What this variable measures]
    \item \textbf{Methodology}: [Econometric or calculation approach]
    \item \textbf{References}: [If using established methodology, cite relevant papers]
    \item \textbf{Frequency and aggregation}: [Temporal frequency and level of aggregation]
\end{itemize}

\textbf{Construction Process:}
\begin{enumerate}
    \item [Step 1: Data sourcing and preparation]
    \item [Step 2: Calculate base variables from raw data]
    \item [Step 3: Perform estimation (e.g., regression, factor extraction)]
    \item [Step 4: Extract key measure of interest from results]
    \item [Step 5: Aggregate to appropriate level]
\end{enumerate}

\subsection*{2.3 Alternative Data Source Processing}

\textbf{[Alternative Data Source: Processing and Integration]}
\begin{itemize}
    \item \textbf{Raw data format}: [Data format and structure as received]
    \item \textbf{Cleaning steps}: [Standardization, deduplication, validation]
    \item \textbf{Variable extraction}: [Specific variables or measures derived]
    \item \textbf{Temporal aggregation}: [How data is transformed to match panel structure]
\end{itemize}

\textbf{Multi-Step Processing Pipeline:}
\begin{enumerate}
    \item [Step 1: Name standardization and disambiguation]
    \item [Step 2: Data retrieval and validation]
    \item [Step 3: Batching or rate-limit management (if applicable)]
    \item [Step 4: Cleaning and quality assurance]
    \item [Step 5: Temporal or geographic aggregation]
    \item [Step 6: Linkage to main analysis sample]
\end{enumerate}

\textbf{Variable Definitions:}
\begin{itemize}
    \item \texttt{[Variable\_Name\_1]}: [Description; transformation if applicable]
    \item \texttt{[Variable\_Name\_2]}: [Description; e.g., log-transformed version]
    \item \texttt{[Variable\_Name\_3]}: [Description; e.g., demeaned or standardized version]
\end{itemize}

\subsection*{2.4 Control and Classification Variables}

\textbf{[Control Variable Category 1: e.g., Firm Characteristics]}
\begin{itemize}
    \item \texttt{[Variable\_1]}: [Definition and source]
    \item \texttt{[Variable\_2]}: [Definition and source]
    \item \texttt{[Variable\_3]}: [Definition and source; e.g., calculated as ratio]
\end{itemize}

\textbf{[Control Variable Category 2: e.g., Risk or Performance Measures]}
\begin{verbatim}
// [Description of calculation approach]
[Formula or pseudocode if complex calculation]
// Aggregation: [temporal or structural aggregation if applicable]
\end{verbatim}

\subsection*{2.5 Classification and Categorical Variables}

\textbf{[Classification Scheme 1: e.g., Percentile-Based Sorting]}
\begin{verbatim}
// [Description of classification logic]
// Calculation by [time period, group, or other dimension]
// Generate indicators for each category or percentile
\end{verbatim}

\textbf{[Classification Scheme 2: e.g., Discrete Categories]}
\begin{itemize}
    \item [Category 1]: [Criteria for assignment]
    \item [Category 2]: [Criteria for assignment]
    \item [Category 3]: [Criteria for assignment]
\end{itemize}

---

\section*{3. Data Integration and Panel Construction}

\subsection*{3.1 Linking Methodology}

\textbf{[Primary Linking Strategy: e.g., GVKEY or Firm Identifier]}
\begin{itemize}
    \item [Primary identifier: description and source]
    \item [Secondary identifiers: alternative matching keys (e.g., CUSIP, CIK, ticker)]
    \item [Special handling for: e.g., corporate actions, ticker changes, delistings]
    \item [Validation approach: method for validating link quality]
\end{itemize}

\textbf{[Geographic Linking: e.g., County or State-Level Matching]}
\begin{itemize}
    \item [Geographic identifier source: e.g., firm headquarters location]
    \item [Geographic code system: e.g., FIPS county codes, state codes]
    \item [Aggregation approach: geographic level used for matching controls]
    \item [Handling missing data: approach for firms with missing location]
\end{itemize}

\textbf{[Additional Linking Dimensions: If Applicable]}
\begin{itemize}
    \item [Linking dimension 1: e.g., industry classification or temporal matching]
    \item [Linking dimension 2: e.g., sector or size-based groupings]
\end{itemize}

\subsection*{3.2 Sample Construction and Filters}

\textbf{Analysis Sample Timeline:}
\begin{itemize}
    \item [Main analysis period: YYYY-YYYY]
    \item [Alternative periods for robustness: YYYY-YYYY (if applicable)]
    \item [Data availability constraints: explanation of why certain variables have different time coverage]
    \item [Justification]: [Why this time period was chosen]
\end{itemize}

\textbf{Sample Inclusion and Exclusion Criteria:}
\begin{verbatim}
// Keep observations meeting inclusion criteria
// [Criteria 1: e.g., valid firm identifiers and year information]
// [Criteria 2: e.g., minimum data requirements per observation]
// [Criteria 3: e.g., industry or geographic restrictions (if applicable)]

// Remove outliers and invalid observations
// [Exclusion 1: e.g., drop observations with missing key variables]
// [Exclusion 2: e.g., drop observations below minimum sample size]
// [Exclusion 3: e.g., winsorize or trim extreme values]
\end{verbatim}

\textbf{Sample Restrictions Justification:}
\begin{itemize}
    \item [Restriction 1: why important for analysis]
    \item [Restriction 2: how it affects sample composition]
    \item [Restriction 3: trade-offs with sample size]
\end{itemize}

\subsection*{3.3 Final Panel Structure}

\textbf{Panel Characteristics:}
\begin{itemize}
    \item \textbf{Panel structure}: [Unit of analysis: firm-year, firm-quarter, firm-month, etc.]
    \item \textbf{Balance}: [Balanced or unbalanced; if unbalanced, explain why firms enter/exit]
    \item \textbf{Final sample size}: [Number of firms, observations, time periods]
    \item \textbf{Data density}: [Percentage non-missing for key variables]
\end{itemize}

\textbf{Variables in Final Dataset:}
\begin{itemize}
    \item [Primary outcome variable(s): count and names]
    \item [Treatment/exposure variable: description]
    \item [Control variables: number and categories]
    \item [Identifiers and time variables: firm ID, year, month, etc.]
    \item [Weights or other adjustments: if applicable]
\end{itemize}

\textbf{Data Quality and Validation:}
\begin{itemize}
    \item [Missing data handling: imputation or deletion strategy]
    \item [Outlier treatment: winsorization levels, dropped categories]
    \item [Validation checks performed: cross-checks with original sources]
    \item [Replication: reproducibility of dataset construction]
\end{itemize}



\end{document} 
%
% To adapt this template:
% 1. Replace [PLACEHOLDER] items with your specific data sources and variables
% 2. Update methodology sections with your actual analysis approach
% 3. Modify variable definitions to match your project requirements
% 4. For integration into Main.tex, remove \documentclass-\begin{document} and \end{document}

\documentclass[11pt]{article}
\usepackage[utf8]{inputenc}
\usepackage[english]{babel}
\usepackage{amsmath}
\usepackage{amssymb}
\usepackage{graphicx}
\usepackage[top=1in, bottom=1in, left=1in, right=1in]{geometry}
\usepackage{hyperref}
\usepackage{longtable}
\usepackage{fancyhdr}
\usepackage{setspace}
\usepackage{float} % For [H] placement of figures/tables

% Custom commands for consistency
\newcommand{\variablefontsize}{\fontsize{9pt}{11pt}\selectfont} % Adjust font size for variable definitions
\newcommand{\tablefontsize}{\fontsize{10pt}{12pt}\selectfont} % Adjust font size for tables

% Page style (no header/footer for simplicity, but can be customized)
\pagestyle{plain}

% Adjust paragraph spacing
\setlength{\parindent}{0pt}
\setlength{\parskip}{1em}

\begin{document}

\section*{Data Sources and Methodology Documentation}
\subsection*{[RESEARCH PROJECT TITLE]}

\subsubsection*{Overview}
This document describes the data sources, variable construction methods, and analytical methodology for the research project examining [RESEARCH QUESTION/HYPOTHESIS].
The documentation provides details on [PRIMARY DATA SOURCE(S)] and how [PRIMARY OUTCOME VARIABLE] is constructed and measured across the sample.

---

\section*{1. Data Sources}

\subsection*{1.1 Primary Data Sources}

\textbf{[Data Source 1: Primary Market/Firm Data]}
\begin{itemize}
    \item [Time frequency and coverage period: e.g., daily/monthly, YYYY-YYYY]
    \item [Key variables provided: e.g., returns, market cap, firm characteristics]
    \item [Additional variables: e.g., location, sector, size indicators]
    \item Used for: [Primary analysis purpose]
    \item Files: [Format and storage location specification]
\end{itemize}

\textbf{[Data Source 2: Secondary Financial Data]}
\begin{itemize}
    \item [Specific data type and frequency: e.g., quarterly accounting data, YYYY-YYYY]
    \item [Firm characteristics available: e.g., financial statements, accounting variables]
    \item [Geographic/identifying information: e.g., headquarters location]
    \item Used for: [Purpose: e.g., control variables, firm linking, sample construction]
\end{itemize}

\textbf{[Data Source 3: Optional Additional Financial Data]}
\begin{itemize}
    \item [Data type and coverage: e.g., analyst forecasts, YYYY-YYYY]
    \item [Specific measures: e.g., forecast dispersion, consensus estimates]
    \item [Coverage and timing: e.g., frequency of measurements, availability]
    \item Used for: [Purpose in analysis]
\end{itemize}

\textbf{[Data Source 4: Factor/Risk Data]}
\begin{itemize}
    \item [Type of factors or risk data: e.g., factor returns, benchmark data]
    \item [Frequency and coverage: e.g., daily returns, YYYY-YYYY]
    \item [Specific factors included: e.g., market, size, value, etc.]
    \item Used for: [Purpose: e.g., risk adjustment, robustness checks]
\end{itemize}

\subsection*{1.2 Treatment and Exogenous Variables}

\textbf{[Primary Treatment/Exposure Variable Source]}
\begin{itemize}
    \item [Description of treatment/exposure: e.g., geographic, temporal, firm-level]
    \item [Measurement level: e.g., county-level, firm-level, industry-level]
    \item [Temporal coverage and frequency: e.g., annual, quarterly]
    \item [Source and documentation: cite original data source]
\end{itemize}

\textbf{[Alternative Data Source: e.g., Additional Measures or Proxies]}
\begin{itemize}
    \item [Description and measurement approach]
    \item [How it relates to primary treatment variable]
    \item [Time periods and frequencies: YYYY-YYYY]
    \item [Aggregation method: e.g., geographic, temporal aggregation]
\end{itemize}

\subsection*{1.3 Control Variables and Demographic Data}

\textbf{[Demographic/Economic Control Data Source]}
\begin{itemize}
    \item [Geographic level: e.g., county-level, state-level]
    \item [Variables available: e.g., population, income, education, employment]
    \item [Frequency and coverage: YYYY-YYYY, annual/quarterly]
    \item [Linking method to analysis sample]
\end{itemize}

\textbf{[Additional Control Variables Source]}
\begin{itemize}
    \item [Type of variables: e.g., economic indicators, market conditions]
    \item [Specific measures: list key variables]
    \item [Coverage period and frequency]
\end{itemize}



---

\section*{2. Variable Construction Methodology}

\subsection*{2.1 Primary Outcome Variable}

\textbf{[Primary Outcome Variable: Name and Definition]}
\begin{itemize}
    \item \textbf{Description}: [Description of what this variable measures and its theoretical importance]
    \item \textbf{Measurement approach}: [How variable is calculated: formula, time-series method, etc.]
    \item \textbf{Data inputs}: [Which source data are used in construction]
    \item \textbf{Frequency}: [Unit of observation: daily, monthly, annual; firm-level, industry-level, etc.]
\end{itemize}

\textbf{Construction Process:}
\begin{enumerate}
    \item [Step 1: Data preparation or filtering]
    \item [Step 2: Raw variable calculation or extraction]
    \item [Step 3: Aggregation or transformation (if applicable)]
    \item [Step 4: Quality checks or validation]
    \item [Step 5: Linking to other datasets (if applicable)]
\end{enumerate}

\subsection*{2.2 Secondary Outcome or Risk Variables}

\textbf{[Secondary Variable Name: Alternative Measure]}
\begin{itemize}
    \item \textbf{Description}: [What this variable measures]
    \item \textbf{Methodology}: [Econometric or calculation approach]
    \item \textbf{References}: [If using established methodology, cite relevant papers]
    \item \textbf{Frequency and aggregation}: [Temporal frequency and level of aggregation]
\end{itemize}

\textbf{Construction Process:}
\begin{enumerate}
    \item [Step 1: Data sourcing and preparation]
    \item [Step 2: Calculate base variables from raw data]
    \item [Step 3: Perform estimation (e.g., regression, factor extraction)]
    \item [Step 4: Extract key measure of interest from results]
    \item [Step 5: Aggregate to appropriate level]
\end{enumerate}

\subsection*{2.3 Alternative Data Source Processing}

\textbf{[Alternative Data Source: Processing and Integration]}
\begin{itemize}
    \item \textbf{Raw data format}: [Data format and structure as received]
    \item \textbf{Cleaning steps}: [Standardization, deduplication, validation]
    \item \textbf{Variable extraction}: [Specific variables or measures derived]
    \item \textbf{Temporal aggregation}: [How data is transformed to match panel structure]
\end{itemize}

\textbf{Multi-Step Processing Pipeline:}
\begin{enumerate}
    \item [Step 1: Name standardization and disambiguation]
    \item [Step 2: Data retrieval and validation]
    \item [Step 3: Batching or rate-limit management (if applicable)]
    \item [Step 4: Cleaning and quality assurance]
    \item [Step 5: Temporal or geographic aggregation]
    \item [Step 6: Linkage to main analysis sample]
\end{enumerate}

\textbf{Variable Definitions:}
\begin{itemize}
    \item \texttt{[Variable\_Name\_1]}: [Description; transformation if applicable]
    \item \texttt{[Variable\_Name\_2]}: [Description; e.g., log-transformed version]
    \item \texttt{[Variable\_Name\_3]}: [Description; e.g., demeaned or standardized version]
\end{itemize}

\subsection*{2.4 Control and Classification Variables}

\textbf{[Control Variable Category 1: e.g., Firm Characteristics]}
\begin{itemize}
    \item \texttt{[Variable\_1]}: [Definition and source]
    \item \texttt{[Variable\_2]}: [Definition and source]
    \item \texttt{[Variable\_3]}: [Definition and source; e.g., calculated as ratio]
\end{itemize}

\textbf{[Control Variable Category 2: e.g., Risk or Performance Measures]}
\begin{verbatim}
// [Description of calculation approach]
[Formula or pseudocode if complex calculation]
// Aggregation: [temporal or structural aggregation if applicable]
\end{verbatim}

\subsection*{2.5 Classification and Categorical Variables}

\textbf{[Classification Scheme 1: e.g., Percentile-Based Sorting]}
\begin{verbatim}
// [Description of classification logic]
// Calculation by [time period, group, or other dimension]
// Generate indicators for each category or percentile
\end{verbatim}

\textbf{[Classification Scheme 2: e.g., Discrete Categories]}
\begin{itemize}
    \item [Category 1]: [Criteria for assignment]
    \item [Category 2]: [Criteria for assignment]
    \item [Category 3]: [Criteria for assignment]
\end{itemize}

---

\section*{3. Data Integration and Panel Construction}

\subsection*{3.1 Linking Methodology}

\textbf{[Primary Linking Strategy: e.g., GVKEY or Firm Identifier]}
\begin{itemize}
    \item [Primary identifier: description and source]
    \item [Secondary identifiers: alternative matching keys (e.g., CUSIP, CIK, ticker)]
    \item [Special handling for: e.g., corporate actions, ticker changes, delistings]
    \item [Validation approach: method for validating link quality]
\end{itemize}

\textbf{[Geographic Linking: e.g., County or State-Level Matching]}
\begin{itemize}
    \item [Geographic identifier source: e.g., firm headquarters location]
    \item [Geographic code system: e.g., FIPS county codes, state codes]
    \item [Aggregation approach: geographic level used for matching controls]
    \item [Handling missing data: approach for firms with missing location]
\end{itemize}

\textbf{[Additional Linking Dimensions: If Applicable]}
\begin{itemize}
    \item [Linking dimension 1: e.g., industry classification or temporal matching]
    \item [Linking dimension 2: e.g., sector or size-based groupings]
\end{itemize}

\subsection*{3.2 Sample Construction and Filters}

\textbf{Analysis Sample Timeline:}
\begin{itemize}
    \item [Main analysis period: YYYY-YYYY]
    \item [Alternative periods for robustness: YYYY-YYYY (if applicable)]
    \item [Data availability constraints: explanation of why certain variables have different time coverage]
    \item [Justification]: [Why this time period was chosen]
\end{itemize}

\textbf{Sample Inclusion and Exclusion Criteria:}
\begin{verbatim}
// Keep observations meeting inclusion criteria
// [Criteria 1: e.g., valid firm identifiers and year information]
// [Criteria 2: e.g., minimum data requirements per observation]
// [Criteria 3: e.g., industry or geographic restrictions (if applicable)]

// Remove outliers and invalid observations
// [Exclusion 1: e.g., drop observations with missing key variables]
// [Exclusion 2: e.g., drop observations below minimum sample size]
// [Exclusion 3: e.g., winsorize or trim extreme values]
\end{verbatim}

\textbf{Sample Restrictions Justification:}
\begin{itemize}
    \item [Restriction 1: why important for analysis]
    \item [Restriction 2: how it affects sample composition]
    \item [Restriction 3: trade-offs with sample size]
\end{itemize}

\subsection*{3.3 Final Panel Structure}

\textbf{Panel Characteristics:}
\begin{itemize}
    \item \textbf{Panel structure}: [Unit of analysis: firm-year, firm-quarter, firm-month, etc.]
    \item \textbf{Balance}: [Balanced or unbalanced; if unbalanced, explain why firms enter/exit]
    \item \textbf{Final sample size}: [Number of firms, observations, time periods]
    \item \textbf{Data density}: [Percentage non-missing for key variables]
\end{itemize}

\textbf{Variables in Final Dataset:}
\begin{itemize}
    \item [Primary outcome variable(s): count and names]
    \item [Treatment/exposure variable: description]
    \item [Control variables: number and categories]
    \item [Identifiers and time variables: firm ID, year, month, etc.]
    \item [Weights or other adjustments: if applicable]
\end{itemize}

\textbf{Data Quality and Validation:}
\begin{itemize}
    \item [Missing data handling: imputation or deletion strategy]
    \item [Outlier treatment: winsorization levels, dropped categories]
    \item [Validation checks performed: cross-checks with original sources]
    \item [Replication: reproducibility of dataset construction]
\end{itemize}



\end{document} 
%
% To adapt this template:
% 1. Replace [PLACEHOLDER] items with your specific data sources and variables
% 2. Update methodology sections with your actual analysis approach
% 3. Modify variable definitions to match your project requirements
% 4. For integration into Main.tex, remove \documentclass-\begin{document} and \end{document}

\documentclass[11pt]{article}
\usepackage[utf8]{inputenc}
\usepackage[english]{babel}
\usepackage{amsmath}
\usepackage{amssymb}
\usepackage{graphicx}
\usepackage[top=1in, bottom=1in, left=1in, right=1in]{geometry}
\usepackage{hyperref}
\usepackage{longtable}
\usepackage{fancyhdr}
\usepackage{setspace}
\usepackage{float} % For [H] placement of figures/tables

% Custom commands for consistency
\newcommand{\variablefontsize}{\fontsize{9pt}{11pt}\selectfont} % Adjust font size for variable definitions
\newcommand{\tablefontsize}{\fontsize{10pt}{12pt}\selectfont} % Adjust font size for tables

% Page style (no header/footer for simplicity, but can be customized)
\pagestyle{plain}

% Adjust paragraph spacing
\setlength{\parindent}{0pt}
\setlength{\parskip}{1em}

\begin{document}

\section*{Data Sources and Methodology Documentation}
\subsection*{[RESEARCH PROJECT TITLE]}

\subsubsection*{Overview}
This document describes the data sources, variable construction methods, and analytical methodology for the research project examining [RESEARCH QUESTION/HYPOTHESIS].
The documentation provides details on [PRIMARY DATA SOURCE(S)] and how [PRIMARY OUTCOME VARIABLE] is constructed and measured across the sample.

---

\section*{1. Data Sources}

\subsection*{1.1 Primary Data Sources}

\textbf{[Data Source 1: Primary Market/Firm Data]}
\begin{itemize}
    \item [Time frequency and coverage period: e.g., daily/monthly, YYYY-YYYY]
    \item [Key variables provided: e.g., returns, market cap, firm characteristics]
    \item [Additional variables: e.g., location, sector, size indicators]
    \item Used for: [Primary analysis purpose]
    \item Files: [Format and storage location specification]
\end{itemize}

\textbf{[Data Source 2: Secondary Financial Data]}
\begin{itemize}
    \item [Specific data type and frequency: e.g., quarterly accounting data, YYYY-YYYY]
    \item [Firm characteristics available: e.g., financial statements, accounting variables]
    \item [Geographic/identifying information: e.g., headquarters location]
    \item Used for: [Purpose: e.g., control variables, firm linking, sample construction]
\end{itemize}

\textbf{[Data Source 3: Optional Additional Financial Data]}
\begin{itemize}
    \item [Data type and coverage: e.g., analyst forecasts, YYYY-YYYY]
    \item [Specific measures: e.g., forecast dispersion, consensus estimates]
    \item [Coverage and timing: e.g., frequency of measurements, availability]
    \item Used for: [Purpose in analysis]
\end{itemize}

\textbf{[Data Source 4: Factor/Risk Data]}
\begin{itemize}
    \item [Type of factors or risk data: e.g., factor returns, benchmark data]
    \item [Frequency and coverage: e.g., daily returns, YYYY-YYYY]
    \item [Specific factors included: e.g., market, size, value, etc.]
    \item Used for: [Purpose: e.g., risk adjustment, robustness checks]
\end{itemize}

\subsection*{1.2 Treatment and Exogenous Variables}

\textbf{[Primary Treatment/Exposure Variable Source]}
\begin{itemize}
    \item [Description of treatment/exposure: e.g., geographic, temporal, firm-level]
    \item [Measurement level: e.g., county-level, firm-level, industry-level]
    \item [Temporal coverage and frequency: e.g., annual, quarterly]
    \item [Source and documentation: cite original data source]
\end{itemize}

\textbf{[Alternative Data Source: e.g., Additional Measures or Proxies]}
\begin{itemize}
    \item [Description and measurement approach]
    \item [How it relates to primary treatment variable]
    \item [Time periods and frequencies: YYYY-YYYY]
    \item [Aggregation method: e.g., geographic, temporal aggregation]
\end{itemize}

\subsection*{1.3 Control Variables and Demographic Data}

\textbf{[Demographic/Economic Control Data Source]}
\begin{itemize}
    \item [Geographic level: e.g., county-level, state-level]
    \item [Variables available: e.g., population, income, education, employment]
    \item [Frequency and coverage: YYYY-YYYY, annual/quarterly]
    \item [Linking method to analysis sample]
\end{itemize}

\textbf{[Additional Control Variables Source]}
\begin{itemize}
    \item [Type of variables: e.g., economic indicators, market conditions]
    \item [Specific measures: list key variables]
    \item [Coverage period and frequency]
\end{itemize}



---

\section*{2. Variable Construction Methodology}

\subsection*{2.1 Primary Outcome Variable}

\textbf{[Primary Outcome Variable: Name and Definition]}
\begin{itemize}
    \item \textbf{Description}: [Description of what this variable measures and its theoretical importance]
    \item \textbf{Measurement approach}: [How variable is calculated: formula, time-series method, etc.]
    \item \textbf{Data inputs}: [Which source data are used in construction]
    \item \textbf{Frequency}: [Unit of observation: daily, monthly, annual; firm-level, industry-level, etc.]
\end{itemize}

\textbf{Construction Process:}
\begin{enumerate}
    \item [Step 1: Data preparation or filtering]
    \item [Step 2: Raw variable calculation or extraction]
    \item [Step 3: Aggregation or transformation (if applicable)]
    \item [Step 4: Quality checks or validation]
    \item [Step 5: Linking to other datasets (if applicable)]
\end{enumerate}

\subsection*{2.2 Secondary Outcome or Risk Variables}

\textbf{[Secondary Variable Name: Alternative Measure]}
\begin{itemize}
    \item \textbf{Description}: [What this variable measures]
    \item \textbf{Methodology}: [Econometric or calculation approach]
    \item \textbf{References}: [If using established methodology, cite relevant papers]
    \item \textbf{Frequency and aggregation}: [Temporal frequency and level of aggregation]
\end{itemize}

\textbf{Construction Process:}
\begin{enumerate}
    \item [Step 1: Data sourcing and preparation]
    \item [Step 2: Calculate base variables from raw data]
    \item [Step 3: Perform estimation (e.g., regression, factor extraction)]
    \item [Step 4: Extract key measure of interest from results]
    \item [Step 5: Aggregate to appropriate level]
\end{enumerate}

\subsection*{2.3 Alternative Data Source Processing}

\textbf{[Alternative Data Source: Processing and Integration]}
\begin{itemize}
    \item \textbf{Raw data format}: [Data format and structure as received]
    \item \textbf{Cleaning steps}: [Standardization, deduplication, validation]
    \item \textbf{Variable extraction}: [Specific variables or measures derived]
    \item \textbf{Temporal aggregation}: [How data is transformed to match panel structure]
\end{itemize}

\textbf{Multi-Step Processing Pipeline:}
\begin{enumerate}
    \item [Step 1: Name standardization and disambiguation]
    \item [Step 2: Data retrieval and validation]
    \item [Step 3: Batching or rate-limit management (if applicable)]
    \item [Step 4: Cleaning and quality assurance]
    \item [Step 5: Temporal or geographic aggregation]
    \item [Step 6: Linkage to main analysis sample]
\end{enumerate}

\textbf{Variable Definitions:}
\begin{itemize}
    \item \texttt{[Variable\_Name\_1]}: [Description; transformation if applicable]
    \item \texttt{[Variable\_Name\_2]}: [Description; e.g., log-transformed version]
    \item \texttt{[Variable\_Name\_3]}: [Description; e.g., demeaned or standardized version]
\end{itemize}

\subsection*{2.4 Control and Classification Variables}

\textbf{[Control Variable Category 1: e.g., Firm Characteristics]}
\begin{itemize}
    \item \texttt{[Variable\_1]}: [Definition and source]
    \item \texttt{[Variable\_2]}: [Definition and source]
    \item \texttt{[Variable\_3]}: [Definition and source; e.g., calculated as ratio]
\end{itemize}

\textbf{[Control Variable Category 2: e.g., Risk or Performance Measures]}
\begin{verbatim}
// [Description of calculation approach]
[Formula or pseudocode if complex calculation]
// Aggregation: [temporal or structural aggregation if applicable]
\end{verbatim}

\subsection*{2.5 Classification and Categorical Variables}

\textbf{[Classification Scheme 1: e.g., Percentile-Based Sorting]}
\begin{verbatim}
// [Description of classification logic]
// Calculation by [time period, group, or other dimension]
// Generate indicators for each category or percentile
\end{verbatim}

\textbf{[Classification Scheme 2: e.g., Discrete Categories]}
\begin{itemize}
    \item [Category 1]: [Criteria for assignment]
    \item [Category 2]: [Criteria for assignment]
    \item [Category 3]: [Criteria for assignment]
\end{itemize}

---

\section*{3. Data Integration and Panel Construction}

\subsection*{3.1 Linking Methodology}

\textbf{[Primary Linking Strategy: e.g., GVKEY or Firm Identifier]}
\begin{itemize}
    \item [Primary identifier: description and source]
    \item [Secondary identifiers: alternative matching keys (e.g., CUSIP, CIK, ticker)]
    \item [Special handling for: e.g., corporate actions, ticker changes, delistings]
    \item [Validation approach: method for validating link quality]
\end{itemize}

\textbf{[Geographic Linking: e.g., County or State-Level Matching]}
\begin{itemize}
    \item [Geographic identifier source: e.g., firm headquarters location]
    \item [Geographic code system: e.g., FIPS county codes, state codes]
    \item [Aggregation approach: geographic level used for matching controls]
    \item [Handling missing data: approach for firms with missing location]
\end{itemize}

\textbf{[Additional Linking Dimensions: If Applicable]}
\begin{itemize}
    \item [Linking dimension 1: e.g., industry classification or temporal matching]
    \item [Linking dimension 2: e.g., sector or size-based groupings]
\end{itemize}

\subsection*{3.2 Sample Construction and Filters}

\textbf{Analysis Sample Timeline:}
\begin{itemize}
    \item [Main analysis period: YYYY-YYYY]
    \item [Alternative periods for robustness: YYYY-YYYY (if applicable)]
    \item [Data availability constraints: explanation of why certain variables have different time coverage]
    \item [Justification]: [Why this time period was chosen]
\end{itemize}

\textbf{Sample Inclusion and Exclusion Criteria:}
\begin{verbatim}
// Keep observations meeting inclusion criteria
// [Criteria 1: e.g., valid firm identifiers and year information]
// [Criteria 2: e.g., minimum data requirements per observation]
// [Criteria 3: e.g., industry or geographic restrictions (if applicable)]

// Remove outliers and invalid observations
// [Exclusion 1: e.g., drop observations with missing key variables]
// [Exclusion 2: e.g., drop observations below minimum sample size]
// [Exclusion 3: e.g., winsorize or trim extreme values]
\end{verbatim}

\textbf{Sample Restrictions Justification:}
\begin{itemize}
    \item [Restriction 1: why important for analysis]
    \item [Restriction 2: how it affects sample composition]
    \item [Restriction 3: trade-offs with sample size]
\end{itemize}

\subsection*{3.3 Final Panel Structure}

\textbf{Panel Characteristics:}
\begin{itemize}
    \item \textbf{Panel structure}: [Unit of analysis: firm-year, firm-quarter, firm-month, etc.]
    \item \textbf{Balance}: [Balanced or unbalanced; if unbalanced, explain why firms enter/exit]
    \item \textbf{Final sample size}: [Number of firms, observations, time periods]
    \item \textbf{Data density}: [Percentage non-missing for key variables]
\end{itemize}

\textbf{Variables in Final Dataset:}
\begin{itemize}
    \item [Primary outcome variable(s): count and names]
    \item [Treatment/exposure variable: description]
    \item [Control variables: number and categories]
    \item [Identifiers and time variables: firm ID, year, month, etc.]
    \item [Weights or other adjustments: if applicable]
\end{itemize}

\textbf{Data Quality and Validation:}
\begin{itemize}
    \item [Missing data handling: imputation or deletion strategy]
    \item [Outlier treatment: winsorization levels, dropped categories]
    \item [Validation checks performed: cross-checks with original sources]
    \item [Replication: reproducibility of dataset construction]
\end{itemize}



\end{document} 
%
% To adapt this template:
% 1. Replace [PLACEHOLDER] items with your specific data sources and variables
% 2. Update methodology sections with your actual analysis approach
% 3. Modify variable definitions to match your project requirements
% 4. For integration into Main.tex, remove \documentclass-\begin{document} and \end{document}

\documentclass[11pt]{article}
\usepackage[utf8]{inputenc}
\usepackage[english]{babel}
\usepackage{amsmath}
\usepackage{amssymb}
\usepackage{graphicx}
\usepackage[top=1in, bottom=1in, left=1in, right=1in]{geometry}
\usepackage{hyperref}
\usepackage{longtable}
\usepackage{fancyhdr}
\usepackage{setspace}
\usepackage{float} % For [H] placement of figures/tables

% Custom commands for consistency
\newcommand{\variablefontsize}{\fontsize{9pt}{11pt}\selectfont} % Adjust font size for variable definitions
\newcommand{\tablefontsize}{\fontsize{10pt}{12pt}\selectfont} % Adjust font size for tables

% Page style (no header/footer for simplicity, but can be customized)
\pagestyle{plain}

% Adjust paragraph spacing
\setlength{\parindent}{0pt}
\setlength{\parskip}{1em}

\begin{document}

\section*{Data Sources and Methodology Documentation}
\subsection*{[RESEARCH PROJECT TITLE]}

\subsubsection*{Overview}
This document describes the data sources, variable construction methods, and analytical methodology for the research project examining [RESEARCH QUESTION/HYPOTHESIS].
The documentation provides details on [PRIMARY DATA SOURCE(S)] and how [PRIMARY OUTCOME VARIABLE] is constructed and measured across the sample.

---

\section*{1. Data Sources}

\subsection*{1.1 Primary Data Sources}

\textbf{[Data Source 1: Primary Market/Firm Data]}
\begin{itemize}
    \item [Time frequency and coverage period: e.g., daily/monthly, YYYY-YYYY]
    \item [Key variables provided: e.g., returns, market cap, firm characteristics]
    \item [Additional variables: e.g., location, sector, size indicators]
    \item Used for: [Primary analysis purpose]
    \item Files: [Format and storage location specification]
\end{itemize}

\textbf{[Data Source 2: Secondary Financial Data]}
\begin{itemize}
    \item [Specific data type and frequency: e.g., quarterly accounting data, YYYY-YYYY]
    \item [Firm characteristics available: e.g., financial statements, accounting variables]
    \item [Geographic/identifying information: e.g., headquarters location]
    \item Used for: [Purpose: e.g., control variables, firm linking, sample construction]
\end{itemize}

\textbf{[Data Source 3: Optional Additional Financial Data]}
\begin{itemize}
    \item [Data type and coverage: e.g., analyst forecasts, YYYY-YYYY]
    \item [Specific measures: e.g., forecast dispersion, consensus estimates]
    \item [Coverage and timing: e.g., frequency of measurements, availability]
    \item Used for: [Purpose in analysis]
\end{itemize}

\textbf{[Data Source 4: Factor/Risk Data]}
\begin{itemize}
    \item [Type of factors or risk data: e.g., factor returns, benchmark data]
    \item [Frequency and coverage: e.g., daily returns, YYYY-YYYY]
    \item [Specific factors included: e.g., market, size, value, etc.]
    \item Used for: [Purpose: e.g., risk adjustment, robustness checks]
\end{itemize}

\subsection*{1.2 Treatment and Exogenous Variables}

\textbf{[Primary Treatment/Exposure Variable Source]}
\begin{itemize}
    \item [Description of treatment/exposure: e.g., geographic, temporal, firm-level]
    \item [Measurement level: e.g., county-level, firm-level, industry-level]
    \item [Temporal coverage and frequency: e.g., annual, quarterly]
    \item [Source and documentation: cite original data source]
\end{itemize}

\textbf{[Alternative Data Source: e.g., Additional Measures or Proxies]}
\begin{itemize}
    \item [Description and measurement approach]
    \item [How it relates to primary treatment variable]
    \item [Time periods and frequencies: YYYY-YYYY]
    \item [Aggregation method: e.g., geographic, temporal aggregation]
\end{itemize}

\subsection*{1.3 Control Variables and Demographic Data}

\textbf{[Demographic/Economic Control Data Source]}
\begin{itemize}
    \item [Geographic level: e.g., county-level, state-level]
    \item [Variables available: e.g., population, income, education, employment]
    \item [Frequency and coverage: YYYY-YYYY, annual/quarterly]
    \item [Linking method to analysis sample]
\end{itemize}

\textbf{[Additional Control Variables Source]}
\begin{itemize}
    \item [Type of variables: e.g., economic indicators, market conditions]
    \item [Specific measures: list key variables]
    \item [Coverage period and frequency]
\end{itemize}



---

\section*{2. Variable Construction Methodology}

\subsection*{2.1 Primary Outcome Variable}

\textbf{[Primary Outcome Variable: Name and Definition]}
\begin{itemize}
    \item \textbf{Description}: [Description of what this variable measures and its theoretical importance]
    \item \textbf{Measurement approach}: [How variable is calculated: formula, time-series method, etc.]
    \item \textbf{Data inputs}: [Which source data are used in construction]
    \item \textbf{Frequency}: [Unit of observation: daily, monthly, annual; firm-level, industry-level, etc.]
\end{itemize}

\textbf{Construction Process:}
\begin{enumerate}
    \item [Step 1: Data preparation or filtering]
    \item [Step 2: Raw variable calculation or extraction]
    \item [Step 3: Aggregation or transformation (if applicable)]
    \item [Step 4: Quality checks or validation]
    \item [Step 5: Linking to other datasets (if applicable)]
\end{enumerate}

\subsection*{2.2 Secondary Outcome or Risk Variables}

\textbf{[Secondary Variable Name: Alternative Measure]}
\begin{itemize}
    \item \textbf{Description}: [What this variable measures]
    \item \textbf{Methodology}: [Econometric or calculation approach]
    \item \textbf{References}: [If using established methodology, cite relevant papers]
    \item \textbf{Frequency and aggregation}: [Temporal frequency and level of aggregation]
\end{itemize}

\textbf{Construction Process:}
\begin{enumerate}
    \item [Step 1: Data sourcing and preparation]
    \item [Step 2: Calculate base variables from raw data]
    \item [Step 3: Perform estimation (e.g., regression, factor extraction)]
    \item [Step 4: Extract key measure of interest from results]
    \item [Step 5: Aggregate to appropriate level]
\end{enumerate}

\subsection*{2.3 Alternative Data Source Processing}

\textbf{[Alternative Data Source: Processing and Integration]}
\begin{itemize}
    \item \textbf{Raw data format}: [Data format and structure as received]
    \item \textbf{Cleaning steps}: [Standardization, deduplication, validation]
    \item \textbf{Variable extraction}: [Specific variables or measures derived]
    \item \textbf{Temporal aggregation}: [How data is transformed to match panel structure]
\end{itemize}

\textbf{Multi-Step Processing Pipeline:}
\begin{enumerate}
    \item [Step 1: Name standardization and disambiguation]
    \item [Step 2: Data retrieval and validation]
    \item [Step 3: Batching or rate-limit management (if applicable)]
    \item [Step 4: Cleaning and quality assurance]
    \item [Step 5: Temporal or geographic aggregation]
    \item [Step 6: Linkage to main analysis sample]
\end{enumerate}

\textbf{Variable Definitions:}
\begin{itemize}
    \item \texttt{[Variable\_Name\_1]}: [Description; transformation if applicable]
    \item \texttt{[Variable\_Name\_2]}: [Description; e.g., log-transformed version]
    \item \texttt{[Variable\_Name\_3]}: [Description; e.g., demeaned or standardized version]
\end{itemize}

\subsection*{2.4 Control and Classification Variables}

\textbf{[Control Variable Category 1: e.g., Firm Characteristics]}
\begin{itemize}
    \item \texttt{[Variable\_1]}: [Definition and source]
    \item \texttt{[Variable\_2]}: [Definition and source]
    \item \texttt{[Variable\_3]}: [Definition and source; e.g., calculated as ratio]
\end{itemize}

\textbf{[Control Variable Category 2: e.g., Risk or Performance Measures]}
\begin{verbatim}
// [Description of calculation approach]
[Formula or pseudocode if complex calculation]
// Aggregation: [temporal or structural aggregation if applicable]
\end{verbatim}

\subsection*{2.5 Classification and Categorical Variables}

\textbf{[Classification Scheme 1: e.g., Percentile-Based Sorting]}
\begin{verbatim}
// [Description of classification logic]
// Calculation by [time period, group, or other dimension]
// Generate indicators for each category or percentile
\end{verbatim}

\textbf{[Classification Scheme 2: e.g., Discrete Categories]}
\begin{itemize}
    \item [Category 1]: [Criteria for assignment]
    \item [Category 2]: [Criteria for assignment]
    \item [Category 3]: [Criteria for assignment]
\end{itemize}

---

\section*{3. Data Integration and Panel Construction}

\subsection*{3.1 Linking Methodology}

\textbf{[Primary Linking Strategy: e.g., GVKEY or Firm Identifier]}
\begin{itemize}
    \item [Primary identifier: description and source]
    \item [Secondary identifiers: alternative matching keys (e.g., CUSIP, CIK, ticker)]
    \item [Special handling for: e.g., corporate actions, ticker changes, delistings]
    \item [Validation approach: method for validating link quality]
\end{itemize}

\textbf{[Geographic Linking: e.g., County or State-Level Matching]}
\begin{itemize}
    \item [Geographic identifier source: e.g., firm headquarters location]
    \item [Geographic code system: e.g., FIPS county codes, state codes]
    \item [Aggregation approach: geographic level used for matching controls]
    \item [Handling missing data: approach for firms with missing location]
\end{itemize}

\textbf{[Additional Linking Dimensions: If Applicable]}
\begin{itemize}
    \item [Linking dimension 1: e.g., industry classification or temporal matching]
    \item [Linking dimension 2: e.g., sector or size-based groupings]
\end{itemize}

\subsection*{3.2 Sample Construction and Filters}

\textbf{Analysis Sample Timeline:}
\begin{itemize}
    \item [Main analysis period: YYYY-YYYY]
    \item [Alternative periods for robustness: YYYY-YYYY (if applicable)]
    \item [Data availability constraints: explanation of why certain variables have different time coverage]
    \item [Justification]: [Why this time period was chosen]
\end{itemize}

\textbf{Sample Inclusion and Exclusion Criteria:}
\begin{verbatim}
// Keep observations meeting inclusion criteria
// [Criteria 1: e.g., valid firm identifiers and year information]
// [Criteria 2: e.g., minimum data requirements per observation]
// [Criteria 3: e.g., industry or geographic restrictions (if applicable)]

// Remove outliers and invalid observations
// [Exclusion 1: e.g., drop observations with missing key variables]
// [Exclusion 2: e.g., drop observations below minimum sample size]
// [Exclusion 3: e.g., winsorize or trim extreme values]
\end{verbatim}

\textbf{Sample Restrictions Justification:}
\begin{itemize}
    \item [Restriction 1: why important for analysis]
    \item [Restriction 2: how it affects sample composition]
    \item [Restriction 3: trade-offs with sample size]
\end{itemize}

\subsection*{3.3 Final Panel Structure}

\textbf{Panel Characteristics:}
\begin{itemize}
    \item \textbf{Panel structure}: [Unit of analysis: firm-year, firm-quarter, firm-month, etc.]
    \item \textbf{Balance}: [Balanced or unbalanced; if unbalanced, explain why firms enter/exit]
    \item \textbf{Final sample size}: [Number of firms, observations, time periods]
    \item \textbf{Data density}: [Percentage non-missing for key variables]
\end{itemize}

\textbf{Variables in Final Dataset:}
\begin{itemize}
    \item [Primary outcome variable(s): count and names]
    \item [Treatment/exposure variable: description]
    \item [Control variables: number and categories]
    \item [Identifiers and time variables: firm ID, year, month, etc.]
    \item [Weights or other adjustments: if applicable]
\end{itemize}

\textbf{Data Quality and Validation:}
\begin{itemize}
    \item [Missing data handling: imputation or deletion strategy]
    \item [Outlier treatment: winsorization levels, dropped categories]
    \item [Validation checks performed: cross-checks with original sources]
    \item [Replication: reproducibility of dataset construction]
\end{itemize}



\end{document} 