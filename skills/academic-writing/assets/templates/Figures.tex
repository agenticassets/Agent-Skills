\setcounter{figure}{0}

% ========================================================================
% FIGURES SECTION - TEMPLATE WITH EXAMPLES
% ========================================================================
% This section contains examples of different figure layouts commonly used
% in academic papers. Follow these templates for consistent formatting:
%
% - Single Figure (portrait orientation)
% - Multi-Panel Figure (landscape orientation with subfloat)
% - Wide Figure (full-width landscape)
%
% See commented examples below for additional formatting options.
% ========================================================================











%%%%%%%%%%  FIGURE 1 - MULTI-PANEL LANDSCAPE  %%%%%%%%%%%%%%%
%%%%%%%%%%  Example of 4-6 panel figure in landscape mode  %%

\begin{landscape}

\begin{figure}[ht!]\centering
\def\sym#1{\ifmmode^{#1}\else\(^{#1}\)\fi}

\caption{\\[Descriptive Figure Title]}
\parbox{\linewidth}{\footnotesize
This figure illustrates [description of what the figure shows].
Panels present [detail about panels A, B, C, etc.].
The [color/pattern/style] indicates [what it represents].
\\}
\label{fig:example_multipanel}    %% LABEL FIGURE FOR REFERENCING IN THE TEXT
\vspace{.5cm}

\subfloat[Panel A: Description]{
  \includegraphics[width=7cm]{Results/Figures/example_figure_1_panelA.pdf}
}
\subfloat[Panel B: Description]{
  \includegraphics[width=7cm]{Results/Figures/example_figure_1_panelB.pdf}
}

\vspace{1cm}
\subfloat[Panel C: Description]{
  \includegraphics[width=7cm]{Results/Figures/example_figure_1_panelC.pdf}
}
\subfloat[Panel D: Description]{
  \includegraphics[width=7cm]{Results/Figures/example_figure_1_panelD.pdf}
}
\subfloat[Panel E: Description]{
  \includegraphics[width=7cm]{Results/Figures/example_figure_1_panelE.pdf}
}
\end{figure}

\end{landscape}

\clearpage





%%%%%%%%%%  FIGURE 2 - SINGLE FIGURE (PORTRAIT)  %%%%%%%%%%%%%
%%%%%%%%%%  Example of standard single figure in portrait   %%

\begin{figure}[htbp]
\centering
\caption{\\[Descriptive Figure Title]}
\parbox{\linewidth}{\footnotesize
This figure shows [description of what the figure displays].
[Additional details about interpretation, methodology, or key findings].
}
\label{fig:example_single}    %% LABEL FIGURE FOR REFERENCING IN THE TEXT
\vspace{.5cm}
\includegraphics[width=0.8\textwidth]{Results/Figures/example_figure_2.pdf}
\end{figure}

\clearpage



%%%%%%%%%%  FIGURE 3 - WIDE LANDSCAPE  %%%%%%%%%%%%%%%%%%%%%%%
%%%%%%%%%%  Example of full-width landscape figure          %%

\begin{landscape}
\begin{figure}[ht!]
\centering
\caption{\\[Descriptive Figure Title for Wide Figure]}
\parbox{\linewidth}{\footnotesize
This figure presents [description of wide landscape figure].
The layout allows [explain advantage of landscape format].
[Additional interpretation details].
}
\label{fig:example_landscape_wide}
\vspace{.5cm}
\includegraphics[width=0.95\textwidth]{Results/Figures/example_figure_3_landscape.pdf}
\end{figure}
\end{landscape}

\clearpage


% ========================================================================
% COMMENTED EXAMPLES - ALTERNATIVE FORMATTING OPTIONS
% ========================================================================
% The examples below show alternative figure layouts and formatting
% strategies that can be used for different types of figures.
% Uncomment and adapt as needed for your specific use case.
% ========================================================================

\begin{comment}

%%%%%%%%%%  EXAMPLE 1 - TIME SERIES FIGURE  %%%%%%%%%%%%%%%%%%
\begin{figure}[ht!]
\centering
\caption{\\[Time Series Title: Variable by Year]}
\parbox{\linewidth}{\footnotesize
This figure reports [variable description] by year from [start year] to [end year].
[Description of trends, patterns, or key observations].
}
\label{fig:example_timeseries}
\vspace{.1cm}
\includegraphics[width=0.85\textwidth]{Results/Figures/example_timeseries.pdf}
\end{figure}
\clearpage


%%%%%%%%%%  EXAMPLE 2 - 3x3 MULTI-PANEL  %%%%%%%%%%%%%%%%%%
\begin{landscape}
\begin{figure}[ht!]
\centering
\caption{\\[Multi-Panel Comparison Title]}
\parbox{\linewidth}{\footnotesize
Panels A-I show [what each panel contains].
[Description of comparison across panels].
}
\label{fig:example_3x3_panels}
\vspace{.5cm}

\subfloat[Panel A]{
  \includegraphics[width=5cm]{Results/Figures/panel_a.pdf}
}
\subfloat[Panel B]{
  \includegraphics[width=5cm]{Results/Figures/panel_b.pdf}
}
\subfloat[Panel C]{
  \includegraphics[width=5cm]{Results/Figures/panel_c.pdf}
}

\vspace{0.8cm}

\subfloat[Panel D]{
  \includegraphics[width=5cm]{Results/Figures/panel_d.pdf}
}
\subfloat[Panel E]{
  \includegraphics[width=5cm]{Results/Figures/panel_e.pdf}
}
\subfloat[Panel F]{
  \includegraphics[width=5cm]{Results/Figures/panel_f.pdf}
}

\vspace{0.8cm}

\subfloat[Panel G]{
  \includegraphics[width=5cm]{Results/Figures/panel_g.pdf}
}
\subfloat[Panel H]{
  \includegraphics[width=5cm]{Results/Figures/panel_h.pdf}
}
\subfloat[Panel I]{
  \includegraphics[width=5cm]{Results/Figures/panel_i.pdf}
}

\end{figure}
\end{landscape}
\clearpage


%%%%%%%%%%  EXAMPLE 3 - SIDE-BY-SIDE COMPARISON  %%%%%%%%%%%%
\begin{figure}[ht!]
\centering
\caption{\\[Comparison Title]}
\parbox{\linewidth}{\footnotesize
Panel A shows [left panel description].
Panel B shows [right panel description].
[Comparative interpretation].
}
\label{fig:example_sidebyside}
\vspace{.5cm}

\subfloat[Panel A: Description]{
  \includegraphics[width=0.45\textwidth]{Results/Figures/example_left.pdf}
}
\hfill
\subfloat[Panel B: Description]{
  \includegraphics[width=0.45\textwidth]{Results/Figures/example_right.pdf}
}

\end{figure}
\clearpage


%%%%%%%%%%  EXAMPLE 4 - EVENT STUDY FIGURE  %%%%%%%%%%%%%%%%%%
\begin{figure}[ht!]
\centering
\caption{\\[Event Study Title: Dynamic Treatment Effects]}
\parbox{\linewidth}{\footnotesize
This figure presents dynamic treatment effects relative to the event window.
The [color/shading] represents [what it indicates].
Shaded region shows [confidence interval/standard error range].
}
\label{fig:example_event_study}
\vspace{.5cm}
\includegraphics[width=0.8\textwidth]{Results/Figures/example_event_study.pdf}
\end{figure}
\clearpage


%%%%%%%%%%  EXAMPLE 5 - HEATMAP FIGURE  %%%%%%%%%%%%%%%%%%%%%
\begin{landscape}
\begin{figure}[ht!]
\centering
\caption{\\[Heatmap Title: Variable Distribution Across Dimensions]}
\parbox{\linewidth}{\footnotesize
This figure presents a heatmap showing [variable description] across [dimensions].
Color intensity indicates [magnitude/direction].
Rows represent [first dimension], columns represent [second dimension].
}
\label{fig:example_heatmap}
\vspace{.5cm}
\includegraphics[width=0.95\textwidth]{Results/Figures/example_heatmap.pdf}
\end{figure}
\end{landscape}
\clearpage



\end{comment}

%%%%%%%%%%  END OF FIGURES  %%%%%%%%%%%%%%%%%%%%%%%%%%%%%%%%%%%%
%%%%%%%%%%%%%%%%%%%%%%%%%%%%%%%%%%%%%%%%%%%%%%%%%%%%%%%%%%%%%%%%
