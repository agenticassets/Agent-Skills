
%%%%%%%%%%  RESET TABLE/FIGURE COUNT WITH "A"  %%%%%%%%%%%%%%%%%
%%%%%%%%%%%%%%%%%%%%%%%%%%%%%%%%%%%%%%%%%%%%%%%%%%%%%%%%%%%%%%%%

\clearpage
	\setcounter{table}{0}
	\renewcommand{\thetable}{A.\arabic{table}}
	\setcounter{figure}{0}
	\renewcommand{\thefigure}{A.\arabic{figure}}
\clearpage

%% ============================================================================
%% INTERNET APPENDIX: ROBUSTNESS CHECKS AND ADDITIONAL SPECIFICATIONS
%% ============================================================================
%%
%% This appendix presents additional robustness specifications, alternative
%% definitions, dynamic analyses, and mechanism tests supporting main results.
%% All tables follow the format of main results with systematic variations in:
%% - Variable definitions and transformations
%% - Dynamic specifications (lags, leads, event windows)
%% - Interaction terms and mechanism tests
%% - Alternative controls and fixed effect structures
%% - Sample variations and subgroup analyses
%%
%% Template Instructions:
%% 1. Replace [TREATMENT] with main independent variable name
%% 2. Replace [OUTCOME] with main dependent variable name
%% 3. Replace [ROBUSTNESS_SPEC] with specification description
%% 4. Replace table input paths: Results/Tables/Table_A[N]_[NAME].tex
%% 5. Adjust table width (\resizebox parameter) as needed for column count
%% ============================================================================

\clearpage

%%%%%%%%%%  TABLE A.1: LAGGED OUTCOME SPECIFICATION  %%%%%%%%%%%
%%%%%%%%%%%%%%%%%%%%%%%%%%%%%%%%%%%%%%%%%%%%%%%%%%%%%%%%%%%%%%%%

\begin{table}[ht!]\centering
\def\sym#1{\ifmmode^{#1}\else\(^{#1}\)\fi}
\caption{[TREATMENT] and [OUTCOME]: Controlling for Lagged Dependent Variable}
\parbox{\linewidth}{\footnotesize
This table presents robustness checks for the baseline specification by including the lagged dependent variable.
The regression model is specified as:
\[
[OUTCOME]_{i,t} = \delta_1\ [TREATMENT]_{i,t} + \delta_2\ [OUTCOME]_{i,t-1} + \gamma X_{i,t} + \alpha_i + \tau_t + \varepsilon_{i,t}
\]
where $[OUTCOME]_{i,t}$ is the main outcome variable for unit $i$ at time $t$, $[TREATMENT]_{i,t}$ is the treatment/policy indicator,
and $[OUTCOME]_{i,t-1}$ is the lagged outcome controlling for persistence.
All specifications include unit fixed effects ($\alpha_i$) and time fixed effects ($\tau_t$).
Column (1) includes baseline variables only.
Column (2) adds [FIRM/UNIT CONTROL VARIABLES].
Column (3) includes [COUNTY/REGIONAL CONTROL VARIABLES].
Standard errors are clustered at the [CLUSTER_VARIABLE].
}
\label{Tab:lagged_outcome_specification}
\vspace{.2cm}
\scriptsize
\resizebox{10cm}{!}{%
%%%%%%%%%%%
\input{../Results/Tables/Table_A1_lagged_outcome_specification.tex}
%%%%%%%%%%%
}
\end{table}
\clearpage

%%%%%%%%%%  TABLE A.2: INTERACTION SPECIFICATION  %%%%%%%%%%%%%
%%%%%%%%%%%%%%%%%%%%%%%%%%%%%%%%%%%%%%%%%%%%%%%%%%%%%%%%%%%%%%%%

\begin{table}[ht!]\centering
\def\sym#1{\ifmmode^{#1}\else\(^{#1}\)\fi}
\caption{[TREATMENT] Effects on [OUTCOME]: Heterogeneous Effects by [MODERATOR]}
\parbox{\linewidth}{\footnotesize
This table examines heterogeneous treatment effects by interacting the main treatment with a moderating variable.
The regression specification is:
\[
[OUTCOME]_{i,t} = \delta_1\ [TREATMENT]_{i,t} \times [MODERATOR]_i + \delta_2\ [TREATMENT]_{i,t} + \delta_3\ [MODERATOR]_i + \gamma X_{i,t} + \alpha_i + \tau_t + \varepsilon_{i,t}
\]
where $[MODERATOR]_i$ captures the moderating mechanism at the unit level, measured at [BASELINE YEAR].
The interaction term tests whether treatment effects vary depending on baseline [MODERATOR] levels.
All specifications include unit fixed effects ($\alpha_i$) and time fixed effects ($\tau_t$).
Column (1) includes baseline variables only.
Column (2) adds [FIRM/UNIT CONTROL VARIABLES].
Column (3) includes [COUNTY/REGIONAL CONTROL VARIABLES].
Standard errors are clustered at the [CLUSTER_VARIABLE].
}
\label{Tab:heterogeneous_treatment_effects}
\vspace{.2cm}
\scriptsize
\resizebox*{\textwidth}{!}{%
%%%%%%%%%%%
\input{../Results/Tables/Table_A2_heterogeneous_effects_by_moderator.tex}
%%%%%%%%%%%
}
\end{table}
\clearpage

%%%%%%%%%%  TABLE A.3: DYNAMIC SPECIFICATION  %%%%%%%%%%%%%%%%%
%%%%%%%%%%%%%%%%%%%%%%%%%%%%%%%%%%%%%%%%%%%%%%%%%%%%%%%%%%%%%%%%

\begin{table}[ht!]\centering
\def\sym#1{\ifmmode^{#1}\else\(^{#1}\)\fi}
\caption{Dynamic Effects of [TREATMENT] on [OUTCOME]: Event Study Specification}
\parbox{\linewidth}{\footnotesize
This table presents a dynamic specification that traces the evolution of treatment effects around the policy adoption year.
The regression model is specified as:
\[
[OUTCOME]_{i,t} = \sum_{k=-K}^{K} \delta_k \cdot \mathbb{1}(t - t^*_i = k) \times [TREATMENT]_i + \gamma X_{i,t} + \alpha_i + \tau_t + \varepsilon_{i,t}
\]
where $\delta_k$ represents the treatment effect $k$ years relative to treatment adoption year $t^*_i$ for unit $i$.
This specification allows the effect to vary over time, revealing the dynamics of adjustment.
The omitted category is [REFERENCE PERIOD, e.g., ``year -2 before treatment''], showing effects relative to the baseline period.
All specifications include unit fixed effects ($\alpha_i$) and time fixed effects ($\tau_t$).
Column (1) includes baseline variables only.
Column (2) adds [FIRM/UNIT CONTROL VARIABLES].
Column (3) includes [COUNTY/REGIONAL CONTROL VARIABLES].
Standard errors are clustered at the [CLUSTER_VARIABLE].
}
\label{Tab:dynamic_treatment_effects}
\vspace{.2cm}
\scriptsize
\resizebox{10cm}{!}{%
%%%%%%%%%%%
\input{../Results/Tables/Table_A3_dynamic_treatment_effects.tex}
%%%%%%%%%%%
}
\end{table}
\clearpage

%% ============================================================================
%% Additional Robustness Specifications Template
%% ============================================================================
%%
%% Below are additional examples showing common robustness checks.
%% Uncomment and customize as needed for your analysis:
%%
%% - Alternative sample restrictions (e.g., excluding outliers, specific subgroups)
%% - Alternative outcome definitions or measurement approaches
%% - Alternative clustering structures or standard error calculations
%% - Placebo tests or falsification exercises
%% - Subsample analyses by firm size, time period, geographic region, etc.
%%
%% Format: Copy the Table A.1 template above, update caption, equation, label,
%% and table input path. Maintain consistent structure across all tables.
%%
%% ============================================================================

% Example template for additional robustness check (uncomment and customize):
%
% \clearpage
%
% %%%%%%%%%%  TABLE A.4: ALTERNATIVE SPECIFICATION  %%%%%%%%%%%%
% %%%%%%%%%%%%%%%%%%%%%%%%%%%%%%%%%%%%%%%%%%%%%%%%%%%%%%%%%%%%%%%%
%
% \begin{table}[ht!]\centering
% \def\sym#1{\ifmmode^{#1}\else\(^{#1}\)\fi}
% \caption{[TREATMENT] and [OUTCOME]: [ROBUSTNESS_SPEC]}
% \parbox{\linewidth}{\footnotesize
% This table presents results for [description of robustness check].
% The regression model is specified as:
% \[
% [OUTCOME]_{i,t} = \delta_1\ [TREATMENT]_{i,t} + [MODIFICATIONS] + \gamma X_{i,t} + \alpha_i + \tau_t + \varepsilon_{i,t}
% \]
% All specifications include unit fixed effects ($\alpha_i$) and time fixed effects ($\tau_t$).
% Standard errors are clustered at the [CLUSTER_VARIABLE].
% }
% \label{Tab:alternative_specification}
% \vspace{.2cm}
% \scriptsize
% \resizebox{10cm}{!}{%
% %%%%%%%%%%%
% \input{../Results/Tables/Table_A4_alternative_specification.tex}
% %%%%%%%%%%%
% }
% \end{table}
% \clearpage
