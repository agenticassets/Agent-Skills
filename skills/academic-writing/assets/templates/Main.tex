%%%%%%%%%%%%%%%%%%%%%%%%%%%%%%%%%%%%%%%%%%%%%%%%%%%%%%%%%%%%%
%% Created by [Your Name]: [Date]
%% Academic Research Template - Based on Finance/Economics Standards
%% Compatible with Journal of Finance, JFE, Real Estate Economics, etc.
%%%%%%%%%%%%%%%%%%%%%%%%%%%%%%%%%%%%%%%%%%%%%%%%%%%%%%%%%%%%%%%

\documentclass[12pt]{article}

% ------------------------------%
%          PACKAGES             %
% ------------------------------%

% Encoding and Fonts
\usepackage[utf8]{inputenc}      % UTF-8 encoding
\usepackage[T1]{fontenc}         % Output font encoding
\usepackage{lmodern}             % Latin Modern font (enhanced version of Computer Modern)
\usepackage{longtable} % Required for tables that span multiple pages.
\usepackage{booktabs}  % For professional quality rules (\toprule, \midrule, \bottomrule).
\usepackage{array}     % For custom column specifications (e.g., ragged-right p-columns).

% Page Layout
\usepackage{geometry}            % Page dimensions and margins
\geometry{
    letterpaper,
    portrait,
    left=1.1in,
    right=1.1in,
    top=1in,
    bottom=1in
}

% Line Spacing
\usepackage{setspace}
\onehalfspacing                   % Set line spacing to 1.5

% Mathematics
\usepackage{amsmath, amssymb, amsthm}    % Math symbols and environments

% Figures and Tables
\usepackage{graphicx}            % Include images
\usepackage{caption}
\captionsetup{justification=centering, format=plain, width=\linewidth}
\usepackage{subcaption}          % Subfigures
\usepackage{booktabs}            % Professional-quality tables
\usepackage{multirow}            % Multi-row cells in tables
\usepackage{array}               % Extended table features
\usepackage{longtable}           % Tables that span multiple pages
\usepackage{pdflscape}           % Landscape pages
\usepackage{adjustbox}           % Adjust box size
\usepackage{siunitx}             % Align numbers in tables

% Bibliography and References
\usepackage[round, longnamesfirst, sort&compress]{natbib}   % Citation management
\bibliographystyle{References/jfe}                          % Use JFE bibliography style
%\bibliographystyle{References/jf}                          % Alternative: Journal of Finance style

% Hyperlinks
\usepackage{hyperref}            % Create hyperlinks
\hypersetup{
    colorlinks=true,
    linkcolor=blue,
    citecolor=blue,
    urlcolor=blue
}

% Colors
\usepackage[dvipsnames]{xcolor}  % Extended colors

% Headers and Footers
\usepackage{fancyhdr}
% Set the page style to 'plain' to have page numbers at the bottom center
\pagestyle{plain}
\fancyhf{}                        % Clear all header and footer fields
%\lhead{[Author's Last Name]}      % Removed author name from header
\rhead{\thepage}                  % Page number on the right
\renewcommand{\headrulewidth}{0pt} % Remove header line
%\renewcommand{\footrulewidth}{0pt} % Remove footer line (already removed by \fancyhf{})

% Section Formatting
\usepackage{titlesec}
\titleformat*{\section}{\centering\large\normalfont\scshape}
\titleformat*{\subsection}{\normalfont\itshape}
\titleformat*{\subsubsection}{\normalfont\itshape}
\titlespacing*{\section}{0pt}{1.0\baselineskip}{0.5\baselineskip}
\titlespacing*{\subsection}{0pt}{0.75\baselineskip}{0.5\baselineskip}
\titlespacing*{\subsubsection}{0pt}{0.5\baselineskip}{0.25\baselineskip}

% Indent first paragraph
\usepackage{indentfirst}

% Miscellaneous
\usepackage{lipsum}              % Dummy text
\usepackage{csquotes}            % Context-sensitive quotation facilities
\usepackage{enumerate}           % Customizable enumerate environment
\usepackage{comment}             % Comment blocks
\usepackage{datetime} % Ensure this package is included
\newdateformat{monthyeardate}{\monthname[\THEMONTH] \THEDAY, \THEYEAR} % Define Month Year format
\usepackage{xcolor}
\usepackage{hyperref}

% Adjust bibliography spacing
\let\OLDthebibliography\thebibliography
\renewcommand\thebibliography[1]{
  \OLDthebibliography{#1}
  \setlength{\parskip}{0pt}
  \setlength{\itemsep}{0pt plus 0.3ex}
}

% Graphics Path
\graphicspath{{../Results/Figures/}}        % Set the graphics path for images from Results directory

% ------------------------------%
%          COMMANDS             %
% ------------------------------%

% Define custom font sizes
\newcommand{\tablefontsize}{\fontsize{9}{11}\selectfont}
\newcommand{\notesize}{\fontsize{9}{11}\selectfont}
\newcommand{\llarge}{\fontsize{13}{15.5}\selectfont}

% Handy shortcut macros
\newcommand{\mc}[1]{\multicolumn{1}{c}{#1}}
\newcommand{\nothing}[1]{}

% Hyphenation penalty
\hyphenpenalty=9500

% ------------------------------%
%       DOCUMENT START          %
% ------------------------------%

\begin{document}

% ------------------------------%
%          TITLE PAGE           %
% ------------------------------%
% TEMPLATE GUIDANCE:
% Replace with your research title, authors, and affiliations
% Follow journal guidelines for title formatting and author information
% Keep title concise and descriptive (typically 10-15 words)

\title{\large \uppercase{[Your Research Title Here]} }
\author{
\textsc{[Author 1 Name]} \thanks{[Institution 1]. Email: \href{mailto:[email1@institution.edu]}{\textcolor{blue}{[email1@institution.edu]}}} 
\and
\textsc{[Author 2 Name]} \thanks{[Institution 2]. Email: \href{mailto:[email2@institution.edu]}{\textcolor{blue}{[email2@institution.edu]}}}
\and
\textsc{[Author 3 Name]} \thanks{[Institution 3]. Email: \href{mailto:[email3@institution.edu]}{\textcolor{blue}{[email3@institution.edu]}}}
}
% \date{\vspace{1cm} \today}
\date{\vspace{1cm} \monthyeardate\today} % Displays Month Year only
\maketitle
\begin{abstract}
% TEMPLATE GUIDANCE FOR ABSTRACT:
% Structure: (1) Research question and motivation, (2) Methodology and identification strategy,
%            (3) Data sources and measures, (4) Key quantitative findings, (5) Contribution to literature
% Length: 150-200 words (check target journal requirements)
% Style: Write in third person, present tense for methods, past tense for findings
% AVOID: Citations, mathematical equations, undefined abbreviations, overly technical language
% DO INCLUDE: Specific quantitative results with magnitudes (e.g., "increases by 15%")

\noindent
This paper examines [RESEARCH QUESTION/PHENOMENON].
We hypothesize that [TREATMENT/INTERVENTION] facilitates [MECHANISM], which affects [OUTCOME VARIABLE].
Using [IDENTIFICATION STRATEGY] and exploiting [SOURCE OF EXOGENOUS VARIATION], we [DESCRIBE EMPIRICAL APPROACH] across [TIME PERIOD AND SAMPLE].
Our analysis incorporates [KEY DATA SOURCES] including [DATA 1] and [DATA 2], combined with [KEY MEASURES/METHODOLOGY].
We find that [MAIN RESULT WITH MAGNITUDE AND SIGNIFICANCE].
These findings contribute to our understanding of [LITERATURE/TOPIC AREA] and have implications for [POLICY/PRACTICE/THEORY].
Results are robust to [ALTERNATIVE SPECIFICATION/SAMPLE].

\vspace{1em}

\noindent \textbf{Keywords:} \textit{[Keyword 1], [Keyword 2], [Keyword 3], [Keyword 4]} \\
\noindent \textbf{JEL Classification:} [G14]; [G12]; [G11]; [G40]; [D83]

\end{abstract}
\thispagestyle{empty}    % Remove header/footer from title page
\clearpage

% % % % % % % % % % % % % % % % % % % % % % % % % % % % % % % % % % % %
%                    TEMPLATE CUSTOMIZATION GUIDE                    %
% % % % % % % % % % % % % % % % % % % % % % % % % % % % % % % % % % % %
%
% BEFORE YOU START WRITING: Read this guide to understand how to customize
% this template for your research project.
%
% STEP 1: CHOOSE YOUR PAPER STRUCTURE
% ────────────────────────────────────
% This template provides TWO organizational structures:
%
% STRUCTURE A (Recommended for tightly integrated work):
%   - Hypotheses, Methods, and Data Sources [ACTIVE - Lines 186-306]
%   - Results, Figures, Appendix, Internet Appendix
%
% STRUCTURE B (Classical academic organization):
%   - Introduction, Literature Review, Data and Methodology, Results,
%     Discussion, Conclusion, Appendix, Internet Appendix
%   - Uncomment sections in the comment block starting at line 308
%
% If using STRUCTURE B:
%   1. Delete or comment out "Hypotheses, Methods, and Data Sources"
%   2. Uncomment the Introduction through Conclusion sections
%   3. Customize the content placeholders
%   4. Keep the Results/Figures/Appendix/Internet Appendix order at the end
%
% STEP 2: CUSTOMIZE PLACEHOLDERS
% ────────────────────────────────
% Replace all [PLACEHOLDER] variables throughout the document:
%
% Title Page:
%   [Your Research Title Here]
%   [Author 1/2/3 Name], [Institution], [email]
%
% Abstract:
%   [RESEARCH QUESTION], [TREATMENT], [MECHANISM], [OUTCOME VARIABLE],
%   [IDENTIFICATION STRATEGY], [DATA SOURCES], [KEY RESULT], [CONTRIBUTION]
%
% Methodology:
%   [TREATMENT VARIABLE], [OUTCOME VARIABLE], [CAUSAL MECHANISM],
%   [SOURCE OF EXOGENOUS VARIATION], [CLUSTERING LEVEL], etc.
%
% Keywords & JEL Classification:
%   [Keywords 1-4] and [JEL Codes]
%
% STEP 3: PREPARE YOUR RESULTS
% ────────────────────────────
% The template automatically imports results from section files:
%   - Writing/sections/Results.tex     (main empirical results)
%   - Writing/sections/Figures.tex     (figures and plots)
%   - Writing/sections/Appendix.tex    (technical appendix)
%   - Writing/sections/Internet_Results.tex (online appendix)
%
% These files use \input{../Results/Tables/[TABLE_NAME].tex} to include
% your auto-generated tables from the analysis pipeline.
%
% STEP 4: COMPILE AND TEST
% ────────────────────────
% After customizing, compile the document:
%   Windows PowerShell: .\Writing\compile_latex.ps1
%   Linux/Mac Bash:     ./Writing/compile_latex.sh
%
% The compilation script handles:
%   - Citation validation
%   - LaTeX compilation (4-step process)
%   - Auxiliary file cleanup
%
% STEP 5: VERIFY INTEGRATION
% ───────────────────────────
% Make sure all \input{} commands can find files:
%   - Results/Tables/ contains your .tex table files
%   - Results/Figures/ contains your .pdf figure files
%   - References/references.bib contains your citations
%
% If compilation fails, check:
%   1. File paths in \input{} and \includegraphics{} commands
%   2. Bibliography entry exists for all cited works
%   3. No special characters or unescaped LaTeX in titles
%
% % % % % % % % % % % % % % % % % % % % % % % % % % % % % % % % % % % %

% Reset page counter and formatting after title page
\setcounter{page}{1}
\normalsize
\setstretch{1.5}

%%%%%%%%%%%%%%%%%%%%%%%%%%%%%%%%%%%%%%%%%%%%%%%%%%%%%%%%%%%%%%%
\section{Hypotheses, Methods, and Data Sources}
%%%%%%%%%%%%%%%%%%%%%%%%%%%%%%%%%%%%%%%%%%%%%%%%%%%%%%%%%%%%%%%

% TEMPLATE GUIDANCE FOR METHODOLOGY SECTION:
% This section introduces your research design and key variables.
% It should answer these questions:
% (1) What is your research question and hypothesis?
% (2) What identification strategy do you use? What is your source of variation?
% (3) What data do you use? How do you measure key variables?
% (4) What is your econometric specification?
%
% STRUCTURE: Research question → Hypothesis → Data → Identification strategy → Econometric model
%
% ALTERNATIVE STRUCTURE: If you prefer traditional sections (Introduction, Literature Review, Data, Methodology),
% see the commented-out sections below this subsection for an alternative template.
% You can uncomment those sections and customize them instead.

\subsection{Research Question and Hypothesis}
% TEMPLATE GUIDANCE:
% 1. State your research question clearly and concisely
% 2. Provide theoretical or empirical motivation
% 3. Explain the proposed causal mechanism
% 4. Preview why this matters (contribution)

This paper investigates whether [RESEARCH QUESTION/PHENOMENON].
We hypothesize that [TREATMENT/INTERVENTION] facilitates [CAUSAL MECHANISM], which affects [OUTCOME VARIABLE].
Specifically, we posit the following channel: [TREATMENT] → [MECHANISM] → [INTERMEDIATE OUTCOME] → [FINAL OUTCOME].
This research question matters because [MOTIVATION/IMPORTANCE FOR THEORY/PRACTICE].

\subsection{Data Sources and Sample Construction}
% TEMPLATE GUIDANCE:
% 1. Identify all data sources and their coverage
% 2. Describe sample selection criteria clearly
% 3. Note any sample restrictions and their justification
% 4. Reference Table 1 (summary statistics) if available

Our analysis uses [KEY DATA SOURCES], including [DATA 1], [DATA 2], and [DATA 3].
[DATA 1] provides [coverage/variables], covering [time period] with [sample size] observations.
We restrict our analysis to [SAMPLE RESTRICTION 1] and [SAMPLE RESTRICTION 2], which yields a final sample of [N OBSERVATIONS, SAMPLE PERIOD].
Summary statistics are presented in Table~\ref{tab:summary_statistics}.

\subsection{Variable Definitions and Measurement}
% TEMPLATE GUIDANCE:
% 1. Define dependent variables clearly
% 2. Define independent/treatment variables
% 3. Explain construction of key variables
% 4. Justify measurement choices by reference to literature

\subsubsection{Dependent Variable: [OUTCOME VARIABLE NAME]}
Our primary dependent variable is [OUTCOME VARIABLE], measured as [MEASUREMENT APPROACH].
Following [KEY REFERENCE PAPERS], we compute [SPECIFIC CALCULATION/FORMULA].
This measure captures [INTERPRETATION/ECONOMIC MEANING].

\subsubsection{Treatment/Independent Variable: [TREATMENT VARIABLE NAME]}
The main independent variable is [TREATMENT VARIABLE], measured as [MEASUREMENT APPROACH].
We employ [TREATMENT VARIATION SOURCE] to identify the causal effect of [TREATMENT].
Specifically, [DESCRIPTION OF VARIATION/SOURCE OF EXOGENOUS VARIATION].

\subsubsection{Control Variables}
The vector $X_{t-1}$ includes lagged control variables: [CONTROL 1] ([DEFINITION/JUSTIFICATION]), [CONTROL 2] ([DEFINITION/JUSTIFICATION]), [CONTROL 3], and [CONTROL 4].
We lag control variables by one period to mitigate simultaneity concerns.

\subsection{Identification Strategy}
% TEMPLATE GUIDANCE:
% 1. Clearly state your identification assumption
% 2. Explain the source of exogenous variation
% 3. Discuss potential confounds and how you address them
% 4. Describe any first-stage or validity tests

The primary challenge to identifying the causal effect of [TREATMENT] on [OUTCOME] is [ENDOGENEITY CONCERN].
We address this through [IDENTIFICATION STRATEGY], exploiting [SOURCE OF EXOGENOUS VARIATION].
Intuitively, [EXPLANATION OF WHY VARIATION IS EXOGENOUS].

[VALIDITY TESTS OR FIRST STAGE EVIDENCE]: We validate this approach by documenting [SUPPORTING EVIDENCE].
This establishes that [TREATMENT] varies significantly across our sample due to [SOURCE OF VARIATION].

\subsection{Econometric Specification}
% TEMPLATE GUIDANCE:
% 1. Present main specification clearly with equation
% 2. Define all variables, subscripts, and notation
% 3. Explain fixed effects and their purpose
% 4. Justify standard error clustering choices
% 5. Note any alternative specifications

Our baseline specification is:
\begin{equation}
\label{eq:main_spec}
[OUTCOME]_{i,t} = \beta_{0} + \beta_{1}[TREATMENT]_{i,t} + X'_{i,t-1}\gamma + \delta_{[UNIT]} + \tau_{[TIME]} + \epsilon_{i,t}
\end{equation}

where $[OUTCOME]_{i,t}$ is the dependent variable for unit [UNIT TYPE] $i$ in period $t$;
$[TREATMENT]_{i,t}$ is the treatment variable;
$X'_{i,t-1}$ is a vector of lagged control variables;
$\delta_{[UNIT]}$ represents [UNIT]-level fixed effects (absorbing [UNIT-LEVEL VARIATION]);
$\tau_{[TIME]}$ represents time fixed effects (absorbing [TIME-LEVEL VARIATION]);
and $\epsilon_{i,t}$ is the error term.

The coefficient of interest is $\beta_{1}$, which captures [INTERPRETATION OF COEFFICIENT].
Standard errors are clustered at the [CLUSTERING LEVEL] to account for [CORRELATION STRUCTURE].

\subsection{Theoretical Framework}
% TEMPLATE GUIDANCE:
% 1. Connect empirical work to theory
% 2. Explain predicted effects under alternative scenarios
% 3. Discuss competing mechanisms or hypotheses
% 4. Provide intuition for expected results and interpretation

Our theoretical framework predicts [NUMBER] possible scenarios:

\begin{enumerate}

\item \textit{Hypothesis 1 - [PRIMARY MECHANISM]:} If [TREATMENT] increases through the mechanism of [MECHANISM 1], then we expect [OUTCOME] to [DIRECTION] by approximately [PREDICTED MAGNITUDE].
This would support the theory that [THEORETICAL INTERPRETATION].

\item \textit{Hypothesis 2 - [ALTERNATIVE MECHANISM]:} If instead [MECHANISM 2] dominates, we would expect [ALTERNATIVE OUTCOME DIRECTION].
This scenario would suggest [ALTERNATIVE THEORETICAL INTERPRETATION].

\end{enumerate}

By examining [MECHANISM TESTS/HETEROGENEOUS EFFECTS], we can distinguish between these scenarios and provide evidence on which mechanism is operative.

\begin{comment}

% % % % % % % % % % % % % % % % % % % % % % % % % % % % % % % % % % % %
% ALTERNATIVE SECTION STRUCTURE (Comment out "Hypotheses, Methods, Data"
% and uncomment sections below if you prefer traditional academic organization)
%
% WHEN TO USE: If your research follows a classical paper structure with
% separate Introduction, Literature Review, Data, Methodology, and Results sections,
% uncomment the sections below and customize them.
%
% HOW TO USE:
% 1. Keep \section{Hypotheses, Methods, and Data Sources} (lines 186-306)
%    OR delete it and uncomment Introduction through Conclusion below
% 2. Uncomment the sections that match your paper structure
% 3. Customize placeholders to your research context
% 4. Keep the Results → Figures → Appendix → Internet Appendix order at the end
% % % % % % % % % % % % % % % % % % % % % % % % % % % % % % % % % % % %

%%%%%%%%%%%%%%%%%%%%%%%%%%%%%%%%%%%%%%%%%%%%%%%%%%%%%%%%%%%%%%%
\section{Introduction}
%%%%%%%%%%%%%%%%%%%%%%%%%%%%%%%%%%%%%%%%%%%%%%%%%%%%%%%%%%%%%%%

% TEMPLATE GUIDANCE FOR INTRODUCTION:
% 1. Open with a compelling motivating example, stylized fact, or puzzle
% 2. State your research question clearly and concisely
% 3. Explain why the question matters (motivation)
% 4. Preview your identification strategy (high level)
% 5. Summarize key findings and economic magnitudes
% 6. Highlight main contributions (3-4 bullet points)
% 7. Provide a roadmap of the paper
%
% Typical structure: Motivation (2-3 paragraphs) → Research question → Identification approach → Findings → Contributions → Roadmap

[Your introduction content here. Begin with motivation and a compelling opening, state your research question clearly, preview your methodology and identification strategy, highlight key contributions, and provide a roadmap of the paper structure.]

%%%%%%%%%%%%%%%%%%%%%%%%%%%%%%%%%%%%%%%%%%%%%%%%%%%%%%%%%%%%%%%
\section{Literature Review}
%%%%%%%%%%%%%%%%%%%%%%%%%%%%%%%%%%%%%%%%%%%%%%%%%%%%%%%%%%%%%%%

% TEMPLATE GUIDANCE FOR LITERATURE REVIEW:
% 1. Organize by THEMES, not chronologically
% 2. Synthesize related papers into coherent narrative
% 3. Identify specific gaps your research addresses
% 4. Position your contribution within the literature landscape
% 5. Connect themes to your research question and hypotheses
%
% Typical structure: Broad theme 1 (2-3 paragraphs) → Theme 2 → Theme 3 → Identified gap → Your contribution

[Review relevant literature. Organize thematically around major research streams, identify specific gaps that your paper addresses, and position your research as a contribution to these literatures.]

%%%%%%%%%%%%%%%%%%%%%%%%%%%%%%%%%%%%%%%%%%%%%%%%%%%%%%%%%%%%%%%
\section{Data and Methodology}
%%%%%%%%%%%%%%%%%%%%%%%%%%%%%%%%%%%%%%%%%%%%%%%%%%%%%%%%%%%%%%%

% TEMPLATE GUIDANCE:
% This section can supplement or replace "Hypotheses, Methods, and Data Sources"
% Include: data sources, sample construction, variable definitions, identification strategy,
% econometric specifications, and justifications for methodological choices

[Describe comprehensive data sources, sample selection procedures, variable construction, and analytical techniques.]

\subsection{Data Sources and Sample Construction}
% TEMPLATE GUIDANCE:
% 1. Document data sources comprehensively
% 2. Specify time periods and coverage
% 3. Describe sample selection criteria
% 4. Note any data quality issues or missing data treatment
% 5. Reference Table 1 for summary statistics

[Detail the data sources, time periods covered, collection methods, sample selection, and any data cleaning or preprocessing steps.]

\subsection{Variable Definitions and Measurement}
% TEMPLATE GUIDANCE:
% 1. Define dependent variable clearly
% 2. Define independent and treatment variables
% 3. Explain construction of each variable
% 4. Justify measurement choices by reference to literature
% 5. Report distributional properties when relevant

[Explain how key variables are constructed, with specific formulas or methodologies.
Detailed variable definitions and extended discussion can be found in Appendix~\ref{appendix:variables}.]

\subsection{Identification Strategy}
% TEMPLATE GUIDANCE:
% 1. Clearly state the identification challenge
% 2. Explain your solution and source of exogenous variation
% 3. Discuss validity assumptions
% 4. Describe any first-stage tests or falsification tests

[Specify your identification strategy, explaining how you address potential endogeneity or confounding. Describe the source of exogenous variation and any tests of validity.]

\subsection{Econometric Specifications}
% TEMPLATE GUIDANCE:
% 1. Present main specification clearly with equation
% 2. Define all variables and notation
% 3. Explain fixed effects and their role
% 4. Justify standard error clustering
% 5. Note alternative specifications

[Specify the econometric models or analytical frameworks employed. Present main specification with equation(s) and explain all notation, fixed effects, and clustering choices.]

Example specification:
\begin{equation}
\label{eq:model}
    Y_{it} = \alpha + \beta X_{it} + \gamma Z_{it} + \delta_i + \tau_t + \varepsilon_{it}
\end{equation}

%%%%%%%%%%%%%%%%%%%%%%%%%%%%%%%%%%%%%%%%%%%%%%%%%%%%%%%%%%%%%%%
\section{Empirical Results}
%%%%%%%%%%%%%%%%%%%%%%%%%%%%%%%%%%%%%%%%%%%%%%%%%%%%%%%%%%%%%%%

% TEMPLATE GUIDANCE:
% Structure results logically: descriptive statistics → main results → robustness → mechanisms
% Use clear table and figure captions with complete information
% Interpret economic significance AND statistical significance
% Reference tables and figures precisely in text

[Present the findings of your analysis systematically, using tables and figures to illustrate key results.]

\subsection{Descriptive Statistics}
% TEMPLATE GUIDANCE:
% 1. Present summary statistics for all key variables
% 2. Compare treatment and control groups if relevant
% 3. Discuss sample composition and representativeness
% 4. Reference Table 1 (main summary statistics table)

[Provide comprehensive summary statistics of main variables. Discuss sample composition and representativeness. Table~\ref{tab:summary_statistics} presents these statistics.]

\subsection{Main Results}
% TEMPLATE GUIDANCE:
% 1. Present results in logical progression
% 2. Start with baseline specification, progressively add controls
% 3. Discuss both statistical and economic significance
% 4. Interpret coefficients in context (not just stars)
% 5. Reference correct table columns and rows

[Discuss the results of your main regression models, presenting coefficients, standard errors, and interpretation.
Table~\ref{tab:main_results} presents the main results.]

\subsection{Robustness and Sensitivity Checks}
% TEMPLATE GUIDANCE:
% 1. Test sensitivity to alternative specifications
% 2. Try alternative samples (subsamples, different time periods)
% 3. Try alternative variable measures
% 4. Include placebo tests or falsification checks
% 5. Address potential confounds explicitly

[Present additional analyses testing robustness of findings. Address alternative explanations and potential confounds.]

%%%%%%%%%%%%%%%%%%%%%%%%%%%%%%%%%%%%%%%%%%%%%%%%%%%%%%%%%%%%%%%
\section{Discussion}
%%%%%%%%%%%%%%%%%%%%%%%%%%%%%%%%%%%%%%%%%%%%%%%%%%%%%%%%%%%%%%%

% TEMPLATE GUIDANCE:
% 1. Interpret results in context of research question
% 2. Relate findings to existing literature
% 3. Discuss economic mechanisms and implications
% 4. Address policy implications if relevant
% 5. Acknowledge limitations honestly
% 6. Suggest directions for future research

[Interpret the results in the context of your research question and the broader literature.
Discuss economic mechanisms, policy implications if relevant, acknowledge limitations, and suggest future research directions.]

%%%%%%%%%%%%%%%%%%%%%%%%%%%%%%%%%%%%%%%%%%%%%%%%%%%%%%%%%%%%%%%
\section{Conclusion}
%%%%%%%%%%%%%%%%%%%%%%%%%%%%%%%%%%%%%%%%%%%%%%%%%%%%%%%%%%%%%%%

% TEMPLATE GUIDANCE:
% 1. Summarize main findings concisely
% 2. Emphasize contributions to theory and/or practice
% 3. Discuss broader implications
% 4. Suggest specific directions for future research
% 5. Keep to 1-2 paragraphs typically

[Summarize the main findings and their significance. Highlight contributions to the literature and practical implications. Suggest future research directions.]

\end{comment}

% ------------------------------%
%          REFERENCES           %
% ------------------------------%

\clearpage
\small    % Set text size for references
\setlength{\bibsep}{0pt}    % Set space between references
\addcontentsline{toc}{section}{\protect\numberline{}References}     % SECTION HEADER
\singlespacing    % Set single spacing for references
\bibliography{References/references}    % Adjust the path to your .bib file
\normalsize    % Reset text size

% % % % % % % % % % % % % % % % % % % % % % % % % % % % % % % % % %
%           MAIN RESULTS SECTION                                 %
% % % % % % % % % % % % % % % % % % % % % % % % % % % % % % % % % %
% TEMPLATE GUIDANCE:
% Main empirical results go here. This section should:
% 1. Present descriptive statistics
% 2. Display main regression results
% 3. Include interpretation of coefficients
% 4. Discuss economic significance
% 5. Reference all tables and figures clearly

\clearpage
% ============================================================================
% TEMPLATE GUIDANCE FOR RESULTS SECTION
% ============================================================================
% This section demonstrates professional academic table formatting and presentation.
% Replace all specific content with your research variables and findings.
% Preserve the LaTeX structure and formatting - it follows journal standards.
% Each table exemplifies best practices for table presentation and captions.
%
% Four example table types are included:
% (1) Summary Statistics - descriptive data for all variables
% (2) Main Results - core regression results with progressive specifications
% (3) Interaction Model - heterogeneous effects testing
% (4) Robustness/Mechanism - alternative specifications and diagnostic tests

% ============================================================================
% TABLE 1: SUMMARY STATISTICS
% ============================================================================

\begin{table}[ht!]\centering
\def\sym#1{\ifmmode^{#1}\else\(^{#1}\)\fi}

% TEMPLATE GUIDANCE FOR SUMMARY STATISTICS TABLE:
% - Include all key analysis variables: treatment, outcomes, controls, fixed effects
% - Order logically: treatment variables first, then primary outcomes, then secondary outcomes, then controls
% - Report: N, mean, median, standard deviation, min, max for each variable
% - Provide clear variable definitions in caption explaining methodology and data construction

\caption{\\Descriptive Statistics: Summary of Key Variables}
\parbox{\linewidth}{\footnotesize
This table presents descriptive statistics for all variables used in the analysis.
The sample spans from [Start Year] to [End Year] and includes [Sample Size] observations of [Unit Type].
[Treatment Variable] represents [brief description of treatment definition and measurement].
[Primary Outcome] measures [description: how is outcome calculated and what does it represent].
[Secondary Outcome] captures [description of second key outcome].
Control variables include firm-level characteristics ([list key firm controls]) and market/geographic characteristics ([list key market controls]).
All monetary variables are measured in [currency unit] and [inflation adjustment if applicable].
Variable definitions and construction details are provided in the Data section and Appendix.
}

\label{tab:summary_statistics}
\vspace{.2cm}

\scriptsize

\resizebox*{\textwidth}{!}{%
\input{../Results/Tables/Table_01_summary_statistics}
}
\end{table}

\clearpage

% ============================================================================
% TABLE 2: MAIN RESULTS
% ============================================================================

\begin{table}[ht!]\centering
\def\sym#1{\ifmmode^{#1}\else\(^{#1}\)\fi}

% TEMPLATE GUIDANCE FOR MAIN RESULTS TABLE:
% - Title should clearly state treatment and outcome variables
% - Equation should match your empirical specification in methodology section
% - Progressive specifications show robustness: baseline → add unit controls → add market controls
% - Report coefficients, standard errors (in parentheses), significance levels (asterisks)
% - Economic magnitude interpretation: what does a one-unit or one-SD increase mean?
% - Fixed effects and standard error clustering should match methodology section

\caption{\\Primary Results: Impact of [Treatment Variable] on [Primary Outcome]}
\parbox{\linewidth}{\footnotesize
This table presents the results of a [identification strategy] analysis examining the effect of [Treatment Variable] on [Primary Outcome].
The baseline regression model is specified as:
\[
[OUTCOME]_{i,t} = \delta_1\ [TREATMENT]_{i,t} + \delta_2\ [OUTCOME]_{i,t-1} + \gamma X_{i,t} + \alpha_i + \alpha_{st} + \epsilon_{i,t}
\]
where $[OUTCOME]_{i,t}$ is [definition of outcome variable: e.g., log of firm performance measure], $[TREATMENT]_{i,t}$ represents [definition of treatment: e.g., indicator or magnitude of treatment], and $[OUTCOME]_{i,t-1}$ is the lagged outcome variable.
The table reports six specifications with progressively increasing controls.
Columns (1)---(2) include baseline variables and [unit]-level fixed effects.
Columns (3)---(4) add firm-level controls including [list key firm controls: e.g., firm size, profitability, leverage].
Columns (5)---(6) further include market-level controls such as [list key market controls: e.g., industry trends, competition].
All specifications include [unit]-level fixed effects ($\alpha_i$) and [industry/geographic]×year fixed effects ($\alpha_{st}$).
Numbers in parentheses are standard errors clustered at the [cluster level].
The sample spans [Start Year] to [End Year] with [Sample Size] [unit]-year observations.
Statistical significance: *** p<0.01, ** p<0.05, * p<0.10.
}

\label{tab:main_results}
\vspace{.2cm}

\scriptsize

\resizebox{0.95\textwidth}{!}{%
\input{../Results/Tables/Table_02_main_results}
}
\end{table}

\clearpage

% ============================================================================
% TABLE 3: HETEROGENEOUS EFFECTS / INTERACTION MODEL
% ============================================================================

\begin{table}[ht!]\centering
\def\sym#1{\ifmmode^{#1}\else\(^{#1}\)\fi}

% TEMPLATE GUIDANCE FOR INTERACTION/HETEROGENEITY TABLE:
% - Use interaction specification to test if effect differs between groups/conditions
% - Alternatively: split sample and show effects for each group in separate columns
% - Clearly define what defines [Group 1] vs [Group 2]
% - Report treatment effect for reference group and interaction term for differential effect
% - Example: H-block effect on firm performance differs based on firm size or market conditions

\caption{\\Heterogeneous Effects: [Treatment Variable] Impact on [Primary Outcome] by [Moderator Variable]}
\parbox{\linewidth}{\footnotesize
This table examines whether the effect of [Treatment Variable] on [Primary Outcome] varies based on [Moderator Variable].
We estimate the interaction model:
\[
[OUTCOME]_{i,t} = \delta_1\ [TREATMENT]_{i,t} + \delta_2\ [MODERATOR]_{i,t} + \delta_3\ ([TREATMENT]_{i,t} \times [MODERATOR]_{i,t}) + \gamma X_{i,t} + \alpha_i + \alpha_{st} + \epsilon_{i,t}
\]
where $[TREATMENT]_{i,t}$ captures the baseline effect for [reference group: e.g., low-moderator firms], while $\delta_3$ represents the differential effect for [high-moderator firms or condition].
[Moderator Variable] is defined as [definition: e.g., median split on firm size or geographic location].
Columns (1)---(2) examine the effect on [Primary Outcome] with and without controls.
Columns (3)---(4) examine effects on [Secondary Outcome].
All specifications include [unit]-level fixed effects and [industry/geographic]×year fixed effects.
Control variables include [list key controls].
Standard errors are clustered at the [cluster level].
The interpretation: when $\delta_3 > 0$, the treatment effect is stronger for [high-moderator group]; when $\delta_3 < 0$, the effect is stronger for [low-moderator group].
}

\label{tab:heterogeneous_effects}
\vspace{.2cm}

\scriptsize

\resizebox{0.95\textwidth}{!}{%
\input{../Results/Tables/Table_03_heterogeneous_effects}
}
\end{table}

\clearpage

% ============================================================================
% TABLE 4: ROBUSTNESS CHECKS / ALTERNATIVE SPECIFICATIONS
% ============================================================================

\begin{table}[ht!]\centering
\def\sym#1{\ifmmode^{#1}\else\(^{#1}\)\fi}

% TEMPLATE GUIDANCE FOR ROBUSTNESS TABLE:
% - Show main specification alongside 3--4 alternative specifications
% - Robustness variations typically include: alternative outcome definitions, alternative clustering, sample restrictions, timing windows, or lagged effects
% - Each column should have a clear label describing the specification variation
% - Use parallel regression structure to highlight robustness (same variables, different specification)

\caption{\\Robustness Checks: Alternative Specifications of [Treatment Variable] Effect on [Primary Outcome]}
\parbox{\linewidth}{\footnotesize
This table presents robustness checks verifying that the main result from Table \ref{tab:main_results} is stable across alternative specifications.
The baseline specification in Column (1) replicates the main result (from Table \ref{tab:main_results} Column X for comparison).
Column (2) uses [alternative outcome measurement: e.g., continuous treatment intensity instead of indicator].
Column (3) restricts the sample to [sample restriction: e.g., large firms, pre-2015 period, or core market].
Column (4) clusters standard errors at [alternative cluster level: e.g., firm instead of firm-year] rather than the baseline [cluster level].
Column (5) includes [additional control or fixed effect: e.g., firm×year fixed effects or industry-specific trends].
All specifications maintain the core regression structure examining the effect of [Treatment Variable] on [Primary Outcome].
The consistency of coefficients across specifications confirms the robustness of the main finding.
Sample size varies slightly across specifications due to [explanation: e.g., sample restrictions or missing control variables].
}

\label{tab:robustness}
\vspace{.2cm}

\scriptsize

\resizebox{0.95\textwidth}{!}{%
\input{../Results/Tables/Table_04_robustness}
}
\end{table}

\clearpage




% % % % % % % % % % % % % % % % % % % % % % % % % % % % % % % % % %
%           FIGURES                                              %
% % % % % % % % % % % % % % % % % % % % % % % % % % % % % % % % % %
% TEMPLATE GUIDANCE:
% Figures and plots go here. Include:
% - Event study plots
% - Time series trends
% - Distribution plots
% - Coefficient comparisons
% All figures should have informative captions and be referenced in text

\clearpage
\setcounter{figure}{0}

% ========================================================================
% FIGURES SECTION - TEMPLATE WITH EXAMPLES
% ========================================================================
% This section contains examples of different figure layouts commonly used
% in academic papers. Follow these templates for consistent formatting:
%
% - Single Figure (portrait orientation)
% - Multi-Panel Figure (landscape orientation with subfloat)
% - Wide Figure (full-width landscape)
%
% See commented examples below for additional formatting options.
% ========================================================================











%%%%%%%%%%  FIGURE 1 - MULTI-PANEL LANDSCAPE  %%%%%%%%%%%%%%%
%%%%%%%%%%  Example of 4-6 panel figure in landscape mode  %%

\begin{landscape}

\begin{figure}[ht!]\centering
\def\sym#1{\ifmmode^{#1}\else\(^{#1}\)\fi}

\caption{\\[Descriptive Figure Title]}
\parbox{\linewidth}{\footnotesize
This figure illustrates [description of what the figure shows].
Panels present [detail about panels A, B, C, etc.].
The [color/pattern/style] indicates [what it represents].
\\}
\label{fig:example_multipanel}    %% LABEL FIGURE FOR REFERENCING IN THE TEXT
\vspace{.5cm}

\subfloat[Panel A: Description]{
  \includegraphics[width=7cm]{Results/Figures/example_figure_1_panelA.pdf}
}
\subfloat[Panel B: Description]{
  \includegraphics[width=7cm]{Results/Figures/example_figure_1_panelB.pdf}
}

\vspace{1cm}
\subfloat[Panel C: Description]{
  \includegraphics[width=7cm]{Results/Figures/example_figure_1_panelC.pdf}
}
\subfloat[Panel D: Description]{
  \includegraphics[width=7cm]{Results/Figures/example_figure_1_panelD.pdf}
}
\subfloat[Panel E: Description]{
  \includegraphics[width=7cm]{Results/Figures/example_figure_1_panelE.pdf}
}
\end{figure}

\end{landscape}

\clearpage





%%%%%%%%%%  FIGURE 2 - SINGLE FIGURE (PORTRAIT)  %%%%%%%%%%%%%
%%%%%%%%%%  Example of standard single figure in portrait   %%

\begin{figure}[htbp]
\centering
\caption{\\[Descriptive Figure Title]}
\parbox{\linewidth}{\footnotesize
This figure shows [description of what the figure displays].
[Additional details about interpretation, methodology, or key findings].
}
\label{fig:example_single}    %% LABEL FIGURE FOR REFERENCING IN THE TEXT
\vspace{.5cm}
\includegraphics[width=0.8\textwidth]{Results/Figures/example_figure_2.pdf}
\end{figure}

\clearpage



%%%%%%%%%%  FIGURE 3 - WIDE LANDSCAPE  %%%%%%%%%%%%%%%%%%%%%%%
%%%%%%%%%%  Example of full-width landscape figure          %%

\begin{landscape}
\begin{figure}[ht!]
\centering
\caption{\\[Descriptive Figure Title for Wide Figure]}
\parbox{\linewidth}{\footnotesize
This figure presents [description of wide landscape figure].
The layout allows [explain advantage of landscape format].
[Additional interpretation details].
}
\label{fig:example_landscape_wide}
\vspace{.5cm}
\includegraphics[width=0.95\textwidth]{Results/Figures/example_figure_3_landscape.pdf}
\end{figure}
\end{landscape}

\clearpage


% ========================================================================
% COMMENTED EXAMPLES - ALTERNATIVE FORMATTING OPTIONS
% ========================================================================
% The examples below show alternative figure layouts and formatting
% strategies that can be used for different types of figures.
% Uncomment and adapt as needed for your specific use case.
% ========================================================================

\begin{comment}

%%%%%%%%%%  EXAMPLE 1 - TIME SERIES FIGURE  %%%%%%%%%%%%%%%%%%
\begin{figure}[ht!]
\centering
\caption{\\[Time Series Title: Variable by Year]}
\parbox{\linewidth}{\footnotesize
This figure reports [variable description] by year from [start year] to [end year].
[Description of trends, patterns, or key observations].
}
\label{fig:example_timeseries}
\vspace{.1cm}
\includegraphics[width=0.85\textwidth]{Results/Figures/example_timeseries.pdf}
\end{figure}
\clearpage


%%%%%%%%%%  EXAMPLE 2 - 3x3 MULTI-PANEL  %%%%%%%%%%%%%%%%%%
\begin{landscape}
\begin{figure}[ht!]
\centering
\caption{\\[Multi-Panel Comparison Title]}
\parbox{\linewidth}{\footnotesize
Panels A-I show [what each panel contains].
[Description of comparison across panels].
}
\label{fig:example_3x3_panels}
\vspace{.5cm}

\subfloat[Panel A]{
  \includegraphics[width=5cm]{Results/Figures/panel_a.pdf}
}
\subfloat[Panel B]{
  \includegraphics[width=5cm]{Results/Figures/panel_b.pdf}
}
\subfloat[Panel C]{
  \includegraphics[width=5cm]{Results/Figures/panel_c.pdf}
}

\vspace{0.8cm}

\subfloat[Panel D]{
  \includegraphics[width=5cm]{Results/Figures/panel_d.pdf}
}
\subfloat[Panel E]{
  \includegraphics[width=5cm]{Results/Figures/panel_e.pdf}
}
\subfloat[Panel F]{
  \includegraphics[width=5cm]{Results/Figures/panel_f.pdf}
}

\vspace{0.8cm}

\subfloat[Panel G]{
  \includegraphics[width=5cm]{Results/Figures/panel_g.pdf}
}
\subfloat[Panel H]{
  \includegraphics[width=5cm]{Results/Figures/panel_h.pdf}
}
\subfloat[Panel I]{
  \includegraphics[width=5cm]{Results/Figures/panel_i.pdf}
}

\end{figure}
\end{landscape}
\clearpage


%%%%%%%%%%  EXAMPLE 3 - SIDE-BY-SIDE COMPARISON  %%%%%%%%%%%%
\begin{figure}[ht!]
\centering
\caption{\\[Comparison Title]}
\parbox{\linewidth}{\footnotesize
Panel A shows [left panel description].
Panel B shows [right panel description].
[Comparative interpretation].
}
\label{fig:example_sidebyside}
\vspace{.5cm}

\subfloat[Panel A: Description]{
  \includegraphics[width=0.45\textwidth]{Results/Figures/example_left.pdf}
}
\hfill
\subfloat[Panel B: Description]{
  \includegraphics[width=0.45\textwidth]{Results/Figures/example_right.pdf}
}

\end{figure}
\clearpage


%%%%%%%%%%  EXAMPLE 4 - EVENT STUDY FIGURE  %%%%%%%%%%%%%%%%%%
\begin{figure}[ht!]
\centering
\caption{\\[Event Study Title: Dynamic Treatment Effects]}
\parbox{\linewidth}{\footnotesize
This figure presents dynamic treatment effects relative to the event window.
The [color/shading] represents [what it indicates].
Shaded region shows [confidence interval/standard error range].
}
\label{fig:example_event_study}
\vspace{.5cm}
\includegraphics[width=0.8\textwidth]{Results/Figures/example_event_study.pdf}
\end{figure}
\clearpage


%%%%%%%%%%  EXAMPLE 5 - HEATMAP FIGURE  %%%%%%%%%%%%%%%%%%%%%
\begin{landscape}
\begin{figure}[ht!]
\centering
\caption{\\[Heatmap Title: Variable Distribution Across Dimensions]}
\parbox{\linewidth}{\footnotesize
This figure presents a heatmap showing [variable description] across [dimensions].
Color intensity indicates [magnitude/direction].
Rows represent [first dimension], columns represent [second dimension].
}
\label{fig:example_heatmap}
\vspace{.5cm}
\includegraphics[width=0.95\textwidth]{Results/Figures/example_heatmap.pdf}
\end{figure}
\end{landscape}
\clearpage



\end{comment}

%%%%%%%%%%  END OF FIGURES  %%%%%%%%%%%%%%%%%%%%%%%%%%%%%%%%%%%%
%%%%%%%%%%%%%%%%%%%%%%%%%%%%%%%%%%%%%%%%%%%%%%%%%%%%%%%%%%%%%%%%


% % % % % % % % % % % % % % % % % % % % % % % % % % % % % % % % % %
%           APPENDIX                                             %
% % % % % % % % % % % % % % % % % % % % % % % % % % % % % % % % % %
% TEMPLATE GUIDANCE:
% Technical appendix includes:
% - Variable definitions and construction
% - Additional robustness checks
% - Methodology details
% - Data quality documentation
% - Supplementary analyses

\clearpage
\section*{Appendix}
\label{appendix:variables}

%%%%%%%%%%%%%%%%%%%%%%%%%%%%%%%%%%%%%%%%%%%%%%%%%%%%%%%%%%%%%%%%
%%%%%%%%%% Variable Definitions %%%%%%%%%%%%%%%%%%%%%%%%%%%%%%
%%%%%%%%%%%%%%%%%%%%%%%%%%%%%%%%%%%%%%%%%%%%%%%%%%%%%%%%%%%%%%%%
%%
%% INSTRUCTIONS: Replace placeholder variable names with your actual variables:
%%   - [OUTCOME_VAR_1], [OUTCOME_VAR_2], etc. → Your dependent variables
%%   - [TREATMENT_VAR], [MODERATOR_VAR], etc. → Your independent variables
%%   - [CONTROL_VAR_1], [CONTROL_VAR_2], etc. → Your control variables
%%
%% Keep the longtable structure and formatting. Structure definitions clearly with:
%%   1. Variable name (left column)
%%   2. Definition + calculation + source (right column)
%%
%% For continuous variables: Include range, calculation method, and source
%% For indicator variables: Explain threshold and coding (0/1)
%% For lagged variables: Specify lag structure
%%

\footnotesize  % Sets a smaller font size for the table
\renewcommand{\arraystretch}{1.5} % Adjusts row spacing for readability within multi-line definitions
\begin{longtable}{>{\raggedright\arraybackslash}p{4cm} p{11cm}}

    \caption*{\\Variable Definitions} \label{tab:variable_description} \\
    \toprule
    \textbf{Variable} & \textbf{Definition and Source} \\
    \midrule
\endfirsthead

    \caption*{\\Variable Definitions (Continued)} \\
    \toprule
    \textbf{Variable} & \textbf{Definition and Source} \\
    \midrule
\endhead

    \midrule
    \multicolumn{2}{r}{\textit{Continued on next page}} \\
\endfoot

    \bottomrule
    \multicolumn{2}{p{\dimexpr\textwidth-2\tabcolsep\relax}} % Adjusted width for notes
    {\textit{Notes:} All firm-level variables are measured [FREQUENCY] unless otherwise specified. Lagged versions (denoted as ``Lag 1'') use the previous period's value. Sample period is [START_YEAR]-[END_YEAR]. All continuous variables are winsorized at the 1st and 99th percentiles to reduce the influence of outliers. Additional variable-specific notes: [INSERT_NOTES].} \\
\endlastfoot

    \multicolumn{2}{l}{\textbf{Main Dependent Variables}} \\
    \midrule

    [OUTCOME_VAR_1] & [Description of how variable is calculated or measured. Include formula if applicable. Example: ``Annualized standard deviation of residuals from [specific model]. Source: [data source].''] \\[1.5ex]

    [OUTCOME_VAR_2] & [Description of measurement methodology and source. Example: ``Percentage change from beginning to end of period. Source: [data source].''] \\[1.5ex]

    \midrule
    \multicolumn{2}{l}{\textbf{Main Independent Variables}} \\
    \midrule

    [TREATMENT_VAR] & [Description of treatment variable, including measurement scale, threshold for indicator variables if applicable, and source. Example: ``Indicator variable equal to 1 if [condition], 0 otherwise. Source: [data source].''] \\[1.5ex]

    [MODERATOR_VAR] & [Description of moderator or interaction variable. Include calculation and source. Example: ``Continuous measure ranging from [min] to [max]. Calculated as [formula]. Source: [data source].''] \\[1.5ex]

    \midrule
    \multicolumn{2}{l}{\textbf{Control Variables}} \\
    \midrule

    [CONTROL_VAR_1] & [Description of control variable. Include measurement and source. Example: ``Natural logarithm of [metric]. Source: [data source].''] \\[1.5ex]

    [CONTROL_VAR_2] & [Description of control variable. Include calculation and source. Example: ``Ratio of [numerator] to [denominator]. Source: [data source].''] \\[1.5ex]

    [CONTROL_VAR_3] & [Description of control variable. Include measurement and source. Example: ``Number of years since [event]. Source: [data source].''] \\

\end{longtable}


% % % % % % % % % % % % % % % % % % % % % % % % % % % % % % % % % %
%   INTERNET APPENDIX (SUPPLEMENTARY RESULTS)                   %
% % % % % % % % % % % % % % % % % % % % % % % % % % % % % % % % % %
% TEMPLATE GUIDANCE:
% Online-only appendix for journals that separate print and online content.
% Includes extensive robustness checks, placebo tests, alternative measures,
% and supplementary analyses referenced in main text.

\clearpage
\section*{Internet Appendix}

%%%%%%%%%%  RESET TABLE/FIGURE COUNT WITH "A"  %%%%%%%%%%%%%%%%%
%%%%%%%%%%%%%%%%%%%%%%%%%%%%%%%%%%%%%%%%%%%%%%%%%%%%%%%%%%%%%%%%

\clearpage
	\setcounter{table}{0}
	\renewcommand{\thetable}{A.\arabic{table}}
	\setcounter{figure}{0}
	\renewcommand{\thefigure}{A.\arabic{figure}}
\clearpage

%% ============================================================================
%% INTERNET APPENDIX: ROBUSTNESS CHECKS AND ADDITIONAL SPECIFICATIONS
%% ============================================================================
%%
%% This appendix presents additional robustness specifications, alternative
%% definitions, dynamic analyses, and mechanism tests supporting main results.
%% All tables follow the format of main results with systematic variations in:
%% - Variable definitions and transformations
%% - Dynamic specifications (lags, leads, event windows)
%% - Interaction terms and mechanism tests
%% - Alternative controls and fixed effect structures
%% - Sample variations and subgroup analyses
%%
%% Template Instructions:
%% 1. Replace [TREATMENT] with main independent variable name
%% 2. Replace [OUTCOME] with main dependent variable name
%% 3. Replace [ROBUSTNESS_SPEC] with specification description
%% 4. Replace table input paths: Results/Tables/Table_A[N]_[NAME].tex
%% 5. Adjust table width (\resizebox parameter) as needed for column count
%% ============================================================================

\clearpage

%%%%%%%%%%  TABLE A.1: LAGGED OUTCOME SPECIFICATION  %%%%%%%%%%%
%%%%%%%%%%%%%%%%%%%%%%%%%%%%%%%%%%%%%%%%%%%%%%%%%%%%%%%%%%%%%%%%

\begin{table}[ht!]\centering
\def\sym#1{\ifmmode^{#1}\else\(^{#1}\)\fi}
\caption{[TREATMENT] and [OUTCOME]: Controlling for Lagged Dependent Variable}
\parbox{\linewidth}{\footnotesize
This table presents robustness checks for the baseline specification by including the lagged dependent variable.
The regression model is specified as:
\[
[OUTCOME]_{i,t} = \delta_1\ [TREATMENT]_{i,t} + \delta_2\ [OUTCOME]_{i,t-1} + \gamma X_{i,t} + \alpha_i + \tau_t + \varepsilon_{i,t}
\]
where $[OUTCOME]_{i,t}$ is the main outcome variable for unit $i$ at time $t$, $[TREATMENT]_{i,t}$ is the treatment/policy indicator,
and $[OUTCOME]_{i,t-1}$ is the lagged outcome controlling for persistence.
All specifications include unit fixed effects ($\alpha_i$) and time fixed effects ($\tau_t$).
Column (1) includes baseline variables only.
Column (2) adds [FIRM/UNIT CONTROL VARIABLES].
Column (3) includes [COUNTY/REGIONAL CONTROL VARIABLES].
Standard errors are clustered at the [CLUSTER_VARIABLE].
}
\label{Tab:lagged_outcome_specification}
\vspace{.2cm}
\scriptsize
\resizebox{10cm}{!}{%
%%%%%%%%%%%
\input{../Results/Tables/Table_A1_lagged_outcome_specification.tex}
%%%%%%%%%%%
}
\end{table}
\clearpage

%%%%%%%%%%  TABLE A.2: INTERACTION SPECIFICATION  %%%%%%%%%%%%%
%%%%%%%%%%%%%%%%%%%%%%%%%%%%%%%%%%%%%%%%%%%%%%%%%%%%%%%%%%%%%%%%

\begin{table}[ht!]\centering
\def\sym#1{\ifmmode^{#1}\else\(^{#1}\)\fi}
\caption{[TREATMENT] Effects on [OUTCOME]: Heterogeneous Effects by [MODERATOR]}
\parbox{\linewidth}{\footnotesize
This table examines heterogeneous treatment effects by interacting the main treatment with a moderating variable.
The regression specification is:
\[
[OUTCOME]_{i,t} = \delta_1\ [TREATMENT]_{i,t} \times [MODERATOR]_i + \delta_2\ [TREATMENT]_{i,t} + \delta_3\ [MODERATOR]_i + \gamma X_{i,t} + \alpha_i + \tau_t + \varepsilon_{i,t}
\]
where $[MODERATOR]_i$ captures the moderating mechanism at the unit level, measured at [BASELINE YEAR].
The interaction term tests whether treatment effects vary depending on baseline [MODERATOR] levels.
All specifications include unit fixed effects ($\alpha_i$) and time fixed effects ($\tau_t$).
Column (1) includes baseline variables only.
Column (2) adds [FIRM/UNIT CONTROL VARIABLES].
Column (3) includes [COUNTY/REGIONAL CONTROL VARIABLES].
Standard errors are clustered at the [CLUSTER_VARIABLE].
}
\label{Tab:heterogeneous_treatment_effects}
\vspace{.2cm}
\scriptsize
\resizebox*{\textwidth}{!}{%
%%%%%%%%%%%
\input{../Results/Tables/Table_A2_heterogeneous_effects_by_moderator.tex}
%%%%%%%%%%%
}
\end{table}
\clearpage

%%%%%%%%%%  TABLE A.3: DYNAMIC SPECIFICATION  %%%%%%%%%%%%%%%%%
%%%%%%%%%%%%%%%%%%%%%%%%%%%%%%%%%%%%%%%%%%%%%%%%%%%%%%%%%%%%%%%%

\begin{table}[ht!]\centering
\def\sym#1{\ifmmode^{#1}\else\(^{#1}\)\fi}
\caption{Dynamic Effects of [TREATMENT] on [OUTCOME]: Event Study Specification}
\parbox{\linewidth}{\footnotesize
This table presents a dynamic specification that traces the evolution of treatment effects around the policy adoption year.
The regression model is specified as:
\[
[OUTCOME]_{i,t} = \sum_{k=-K}^{K} \delta_k \cdot \mathbb{1}(t - t^*_i = k) \times [TREATMENT]_i + \gamma X_{i,t} + \alpha_i + \tau_t + \varepsilon_{i,t}
\]
where $\delta_k$ represents the treatment effect $k$ years relative to treatment adoption year $t^*_i$ for unit $i$.
This specification allows the effect to vary over time, revealing the dynamics of adjustment.
The omitted category is [REFERENCE PERIOD, e.g., ``year -2 before treatment''], showing effects relative to the baseline period.
All specifications include unit fixed effects ($\alpha_i$) and time fixed effects ($\tau_t$).
Column (1) includes baseline variables only.
Column (2) adds [FIRM/UNIT CONTROL VARIABLES].
Column (3) includes [COUNTY/REGIONAL CONTROL VARIABLES].
Standard errors are clustered at the [CLUSTER_VARIABLE].
}
\label{Tab:dynamic_treatment_effects}
\vspace{.2cm}
\scriptsize
\resizebox{10cm}{!}{%
%%%%%%%%%%%
\input{../Results/Tables/Table_A3_dynamic_treatment_effects.tex}
%%%%%%%%%%%
}
\end{table}
\clearpage

%% ============================================================================
%% Additional Robustness Specifications Template
%% ============================================================================
%%
%% Below are additional examples showing common robustness checks.
%% Uncomment and customize as needed for your analysis:
%%
%% - Alternative sample restrictions (e.g., excluding outliers, specific subgroups)
%% - Alternative outcome definitions or measurement approaches
%% - Alternative clustering structures or standard error calculations
%% - Placebo tests or falsification exercises
%% - Subsample analyses by firm size, time period, geographic region, etc.
%%
%% Format: Copy the Table A.1 template above, update caption, equation, label,
%% and table input path. Maintain consistent structure across all tables.
%%
%% ============================================================================

% Example template for additional robustness check (uncomment and customize):
%
% \clearpage
%
% %%%%%%%%%%  TABLE A.4: ALTERNATIVE SPECIFICATION  %%%%%%%%%%%%
% %%%%%%%%%%%%%%%%%%%%%%%%%%%%%%%%%%%%%%%%%%%%%%%%%%%%%%%%%%%%%%%%
%
% \begin{table}[ht!]\centering
% \def\sym#1{\ifmmode^{#1}\else\(^{#1}\)\fi}
% \caption{[TREATMENT] and [OUTCOME]: [ROBUSTNESS_SPEC]}
% \parbox{\linewidth}{\footnotesize
% This table presents results for [description of robustness check].
% The regression model is specified as:
% \[
% [OUTCOME]_{i,t} = \delta_1\ [TREATMENT]_{i,t} + [MODIFICATIONS] + \gamma X_{i,t} + \alpha_i + \tau_t + \varepsilon_{i,t}
% \]
% All specifications include unit fixed effects ($\alpha_i$) and time fixed effects ($\tau_t$).
% Standard errors are clustered at the [CLUSTER_VARIABLE].
% }
% \label{Tab:alternative_specification}
% \vspace{.2cm}
% \scriptsize
% \resizebox{10cm}{!}{%
% %%%%%%%%%%%
% \input{../Results/Tables/Table_A4_alternative_specification.tex}
% %%%%%%%%%%%
% }
% \end{table}
% \clearpage


% ------------------------------%
%           END DOCUMENT        %
% ------------------------------%

\end{document}
